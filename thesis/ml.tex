\chapter{Machine Learning}

This section outlines the basic methodology followed in machine
learning approaches to Natural Language Processing. I will briefly
discuss machine learning from a general point of view and then present
supervised machine learning in more detail using linear regression as
example. I will then present on the different kinds of classifiers
applied in NLP; both unstructured and structured.

\subsection{Overview}
%\begin{itemize}
%\item Rule-based vs. machine learning approaches.
%\item The basic machine learning tasks: classification and clustering.
%\item The basic machine learning approaches: supervised and
%  unsupervised learning.
%\end{itemize}

Approaches to Natural Language Processing can be broadly divided into
two groups: rule-based and machine learning approaches. Rule-based
approaches utilize hand crafted rules for tasks such as text labeling
or machine translation. Machine learning approaches solve the same set
of problems using data driven (usually statistically motivated) models
and samples from the problem domain, which are used to estimate model
parameters.

This thesis focuses on extending machine learning techniques to the
domain of morphologically complex languages, where they have been
applied less frequent than for morphologically more straightforward
languages like English. I want to emphasize that the thesis should not
be seen as an attack against the rule-based paradigm for Natural
Language Processing. It is simply an investigation into extension of
machine learning techniques.

Machine learning and hand crafted rules have their respective merits
and short-comings. Rule-based methods can often give much better
results and so on.

Machine Learning is can be divided into two fields supervised and
unsupervised Machine Learning. Supervised machine translation uses a
training material with hand crafted annotations. An example would be a
set of images with associated key words, where the task is to build a
system which can associate previously unseen images with appropriate
key words.

Supervised machine learning is typically employed for tasks such as
labeling (that is classification) and translation. Typical examples of
tasks include Part-of-Speech labeling and speech recognition where
annotations consist of POS labels for the words in some text and
written sentences corresponding to an acoustic speech signal
respectively.

In contrast to supervised machine translation, unsupervised approaches
exclusively utilize unannotated data. Unsupervised machine learning is
most often used for various kinds of clustering tasks, that is
grouping sets of examples into subsets of similar examples. 

In Natural Language processing, supervised learning approaches are
most often used (for example ...). Unsupervised approaches, however,
can also be useful for e.g. exploratory data analysis (...).

This thesis focuses on supervised machine learning.

\section{Supervised Learning}
%\begin{itemize}
%\item Example: linear regression.
%\item Estimation from a sample.
%\item Convexity, smoothness.
%\item Regularization.
%\item Methodology: Training, development and test sets. 
%\item Significance testing.
%\end{itemize}

In this section, I will illustrate the key concepts and techniques in
supervised Machine Learning using the simple example of linear regression. I will explain the plain linear regression model and show how it can be fitted using training data. I will then briefly present ridge regression which is a regularized version of linear regression.

I chose linear regression as example because it is a simple model yet
can be used to illustrate data driven techniques. Moreover, the model
has several tractable properties such as smoothness and
convexity. Additionally, it can be seen as the simplest example of a
linear classifier which is a category of models encompassing

\subsection{Basic Concepts}

\paragraph{Linear Regression} As a simple example, imagine a person
called Jill who is a real estate agent. She is interested in
constructing an application, for use by prospective clients, which
would give rough estimates for the selling price of a property. Jill
knows that a large number of factors affect housing prices but there
are a few very robust quantifiable predictors of price that are easy
to measure.

Jill decides to base the model on the following predictors:
\begin{enumerate}
\item The living area.
\item The number of rooms.
\item The number of bathrooms.
\item Size of the yard.
\item Distance of the house from the city center.
\item Age of the house.
\item Amount of time since the last major renovation.
\end{enumerate}

Jill decides to use the simplest model which seems reasonable. This
model is {\it linear regression} which models {\it the dependent
  variable} house price as a linear combination of the independent
variables listed above and parameter values in $\R$. The linear
regression model is probably not accurate. It fails in several
regards. For example, increasing age of the house probably reduces the
price up to a point but very old houses can in fact be more expensive
then newly built houses especially if they have been renovated lately.
Although, the linear model is unlikely to be entirely accurate, Jill is
happy with it because the intention is just to give a ball park
estimate of the price for the average prospective client.

To formalize the linear regression model, let us call the dependent
variable price $y$ and each of the independent variables living area,
number of rooms and so on $x_i$. Given a vector $x = (x[1], ...,
x[n])^\top \in \R^n$, which combines the independent
variables\footnote{In reality, each of the predictors would probably
  be transformed to give all of them the same average and
  variance. Although this ii not required in theory, it tends to give
  a better model \cite{someone}.}, and a parameter vector $\theta \in
\R^n$ the linear regression model is given in Equation
\ref{eq:linreg}.

\begin{equation}
y(x\parcond \theta) = x^\top\theta\label{eq:linreg}
\end{equation}

Two questions immediately arise: How to compute the price given
parameters and predictors and how to compute the parameter vector
$\theta$. These questions are common for all supervised learning
problems also when using other models than the linear regression
model.

\paragraph{Inference} The first question concerns {\it inference},
that is finding the most likely value for the dependent variable given
the independent variables. In the case of linear regression, the
answer to this question is straightforward. To compute the price,
simply perform the inner product in Equation \ref{eq:linreg}. The
question is, however, not entirely settled because one might also ask
for example how close to the actual price the estimate $y$ is likely
to be. A related question would be to provide an upper and lower bound
for the price so that the actual price is very likely to be inside the
provided bounds. To answer these questions, one would have to model
the expected error.

Inference is very easy and also efficient in the case of linear
regression. With more complex model such as structured graphical
models used in this thesis, it can however be an algorithmically and
computationally more challenging problem. The principle is still the
same: Find the $y$ which is most likely given the input parameters.

\paragraph{Estimation} The second question concerns {\it estimation of
  model parameters} and it is more complex than the question of
inference. First of all, Jill needs training data. She also needs to
decide upon an {\it estimator}, that is, a method for estimating the
parameters $\theta$.

In the case of house price prediction, Jill can simply use data about
houses she has brokered in the past. She decides to use a training
data set $\mathcal{D} = \{(x_1, y_1), ..., (x_T, y_T)\}$, where each
$x_t = (x_t[1] ... x_t[n])$ is a vector of independent variable values
(living area, age of the house and so on) and $y_t$ is the dependent
variable value, that is the final selling price of the house. Now Jill
needs to make a choice. How many training examples $(x_t, y_t)$ does
she need? The common wisdom is that more data is always better,
however, this has bearing on how the parameters need to be
estimated. In any case, the number of data points in the training data
should ideally be higher than the number of parameters that need to be
estimated. When one cannot accomplish this, one encounters the so
called {\it data sparsity problem}.

\paragraph{Data Sparsity} Whereas getting sufficient training data is
fairly easy in the case of housing prices, it is vastly more difficult
to accomplish with more complicated models in natural language
processing. Therefore, one central question in this thesis is how to
counteract {\it data sparsity}.

\paragraph{Cost Functions} The objective in estimation is to find a
parameter vector $\theta$ which in some sense minimizes the error of
the house price predictions $y(x_t\parcond \theta)$ when compared to
the actual realized house prices $y_t$ in the training data. The usual
minimization criterion used with linear regression is the least square
sum criterion given in Equation \ref{eq:lss}. It is minimized by a
parameter vector $\theta$ which gives as small square errors $|y_t -
y(x_t \parcond \theta)|^2$ as possible.

\begin{equation}
\theta = \argmin_{\theta' \in \R^n} \sum_{x_t \in \mathcal{D}} | y_t - y(x_t\parcond \theta')|^2\label{eq:lss}
\end{equation}

The square sum is an example of a {\it cost function} (also
called the objective function). A cost function assigns a non-negative
real cost for each parameter vector. Using the concept of cost
function, the objective of estimation can be reformulated: Find the
parameter vector $\theta$ that minimizes the cost of the training
data.

\paragraph{The Exact Solution} In the case of linear regression, there
is a well known exact solution for $\theta$ which utilizes linear
algebra. The solution is given in Equation \ref{eq:lss-exact}. The
matrix $X \in \R^T_n$ is defined by $X_{t,i} = x_t[i]$ and its
More-Pennrose pseudoinverse $X^+ \in \R^n_T$ is defined as $X^+ =
(X^\top X)^{-1}X^\top$. The vector $Y \in R^T$ is simply the vector of
realized house prices $y_t$. The solution $\theta$ exists only when
none of the independent variables are linear combinations of each
other in the training data.

\begin{equation}
\theta = X^+ Y \label{eq:lss-exact}
\end{equation}

\paragraph{Iterative Estimation} Although the linear regression model
is simple enough so that it can be estimated exactly, the same does
not hold for most more complex models such as the Conditional Random
Field investigated in this thesis. Moreover, the exact solution might
often not be the one that is desired. Often the training data is quite
sparse, that is there is too little of it compared to the amount of
parameters that need to be estimated. Therefore, the model may {\it
  over-fit} the training data and fail to generalize well to examples
not included in the training data. 

\paragraph{Regularization} Due to the problem of over-fitting, a family
of heuristic techniques called {\it regularization} is often
employed. They aim to transform the original problem in a way which
will penalize both deviance from the gold standard and ``complexity''
of the solution $\theta$. Regularization can be seen to convey the
same idea as Occam's Maxim which states that a simpler explanation for
a phenomenon should be preferred when compared to a more complex
explanation yielding equivalent results. Of course, this does not
explain what is meant by a ``complex'' parameter vector $\theta$.

To illustrate simple and complex parameter vectors, examine a case of
linear regression where the dependent variable $y$ and the predictors
$x_i$ have mean $0$ and variance $1$ in the training data. This may
seem restrictive but in fact any linear regression problem can easily
be transformed into this form by applying an affine transformation $z
\mapsto az - b$. When doing inference, this affine transformation can
simply be reversed by applying $z \mapsto a^{-1} (z + b)$. The
simplest parameter vector $\theta$ is clearly the zero vector $\theta
= (0 ... 0)^\top$. It corresponds to the hypothesis that the
predictors $x_i$ have no effect on the dependent variable
$y$. According to this hypothesis, the prediction is always the
average of the dependent variable values in the training data.

The zero solution to a linear regression problem is simple but also
totally biased. Because we are assuming that the independent variables
$x_i$ explain the dependent variable $y$, a model that completely
disregards them is unlikely to give a good fit to the training
data. By introducing a regularization term into the cost function, we
can however encourage simple solutions while at the same time also
preferring solutions that give a good fit. There are several ways to
accomplish this but the most commonly used are so called $L_1$ and
$L_2$ regularization. These are general regularization methods that
are employed in many models in machine learning.

The $L_1$ regularized cost function for linear regression is given in
Equation \ref{eq:lss-l1}. $L_1$ regularization, also called LASSO
regularization \cite{somone}, enforces solutions where many of the
parameter values are $0$. These are also called sparse parameters. It
is suitable in the situation where the model is over specified, that
is, many of the predictors might not be necessary for good prediction. The expression is a sum 

\begin{equation}
\theta = \argmin_{\theta' \in \R^n} \sum_{x_t \in \mathcal{D}} | y_t - y(x_t\parcond \theta')|^2 + \lambda \sum_i |\theta[i]|\label{eq:lss-l1}
\end{equation}

The $L_2$ regularized cost function is given in \ref{eq:lss-l2}. $L_2$
regularization is also called Tikhonov regularization. In contrast to
$L_1$ regularization, it directly prefers solutions with small norm. A
linear regression model with Tikhonov regularization is called a ridge
regression model.

\begin{equation}
\theta = \argmin_{\theta' \in \R^n} \sum_{x_t \in \mathcal{D}} | y_t - y(x_t\parcond \theta')|^2 + \lambda \|\theta\|^2 = \argmin_{\theta' \in \R^n} \sum_{x_t \in \mathcal{D}} | y_t - y(x_t\parcond \theta')|^2 + \lambda \sum_i |\theta[i]|^2\label{eq:lss-l2}
\end{equation}

The coefficient $\lambda \in \R^+$ is called the {\it
  regularizer}. The regularizer determines the degree to which model
fit and simplicity affect the cost. A higher $\lambda$ will increase
the cost for complex models more than a lower one. When $\lambda$
increases, the optimal parameter vector $\theta$ approaches the zero
vector and when it decreases $\theta$ approaches the parameters that
fit the training data as closely as possible. This is called
under-fitting.

The regularizer is a so called {\it hyper-parameter} of the
regularized liner regression model. It is easy to see that increasing
$\lambda$ will automatically increase the cost. Therefore, there is no
direct way to estimate its correct magnitude simply using the training
data. Instead {\it held-out data} can be used. Held-out data is
labeled data that is not used directly for estimating model
parameters. If the model over-fits the training data, that is
generalizes poorly to unseen examples, the held-out data will have a
high cost. However, it will also have a high cost if the model
under-fits, that is, performs poorly on all data. Held-out data can
therefore be used to find an optimal values for the regularizer
$\lambda$. Often one tries several potential values and chooses the
one that minimizes the cost of the held-out data. Usually, one uses
the unregularized cost function for the held-out data.

\paragraph{Iterative Estimation} Regularization is an additional reason for
introducing iterative estimation methods instead of exact
estimation. There are several choices of regularizers and some of them
might not result in optimization problems that have closed form
solutions\footnote{In the case of regression, non-linear {\it kernel
    functions} also result in optimization problems that don't have a
  closed form solution. I will elaborate this when discussing logistic
  regression.}. When an exact solution cannot be computed, or it is
undesirable, numerical methods can be used to estimate model
parameters.

\begin{itemize}
\item Smoothness of cost function.
\item Convexity.
\item Compactness. Not getting stuck at local optima does not
  guarantee that a global optimum can be found.
\item Stochastic Gradient Descent.
\item Quasi Newton Methods.
\end{itemize}

\section{Classifiers}
\begin{itemize}
\item Naïve Bayes' classifier.
\item Generative and Discriminative Classifiers.
\item Linear classifiers: logistic classifier, perceptron classifier \citep{Freund1999},
  SVM \citep{Cortes1995}.
\end{itemize}

\section{Structured Classifiers}
\begin{itemize}
\item The idea of chained stochastic processes.
\item Markov chain.
\item HMM
\item MEMM
\item CRF
\end{itemize}
