\chapter{Introduction}
\label{ch:intro}

The topic of this thesis is morphological tagging. A morphological
tagger is a piece of computer software which provides complete
morphological descriptions of the words in a sentence as seen in
Figure \ref{fig:mt-example}. The main application of morphological
taggers is as a pre-processing step for some other natural language
application, for example a syntactic parser.

The task of morphological tagging is closely related to part-of-speech
tagging (POS tagging) where a coarse morphological labels, such as
noun or verb, is given for each word in sentence. In contrast to a POS
tagger, a morphological tagger, however, provides more detailed
morphological descriptions accompanied by lemmas. This is necessary
when processing {\it morphologically rich languages}, which utilize
the internal structure of words to encode a variety of structural and
semantic information. For these languages, a plain part-of-speech does
not provide enough information to enable successful further
processing.

\begin{figure}[!htb]
\begin{center}
\begin{tabular}{|l|l|l|l|l|l|l|}
\hline
Article+Indef & Noun+Sg & Verb+Pres+3sg & Prep & Article+Def & Noun+Sg & .\\  
\hline
a & dog & sleep & on & the & mat & .\\
\hline
A & dog & sleeps & on & the & mat & .\\
\hline
\end{tabular}
\end{center}
\end{figure}

At first glance, the task of assigning morphological descriptions, or
{\it morphological labels}, seems very easy. Simply form a list of
common word forms and their morphological labels and look up words in
the list when tagging unseen sentences. Unfortunately, this approach
fails because of the following reasons.

\begin{enumerate}
\item A single word form can get several morphological labels
  depending on context. For example ``dog'' and ``man'' can be both
  nouns and verbs in English.
\item For morphologically rich languages, it is impossible to form a
  list of common word forms which would have sufficient coverage (say,
  higher than 95\%) on unseen text.
\end{enumerate}

Due to the reasons mentioned above, a high accuracy morphological
tagger must model the context of a word in order to be able to
disambiguate between its alternative analyses. Moreover, it has to
model the structure of words in order to be able to assign
morphological labels for previous unseen word forms.

This thesis presents work on building morphological taggers for
morphologically rich languages, in particular Finnish which is the
native language of the author. The thesis focuses on data-driven
methods which utilize manually prepared training corpora and machine
learning techniques to derive tagger models.

\section{Motivation}
Data-driven methods have dominated the field of natural language
processing (NLP). Although these methods have been applied to
virtually all language processing tasks, research has predominantly focused
on the English language. For many smaller languages such as Finnish,
statistical methods have not been applied to the same extent. This is
probably due to the fact that large training corpora required by
supervised data-driven methods are available for very few languages.

The lack of statistical work on NLP for languages besides English is a
problem because the languages of the word differ substantially with
regard to syntax, morphology, phonology and orthography. These
differences have very real consequences for the design of NLP
systems. Therefore, it is impossible to make general claims about
language processing without testing these claims on other languages
than English.

This thesis presents work that focuses on data-driven methods for
morphological tagging of Finnish which is the native language of the
author. Finnish and English share many characteristics but also differ
in many respects. Both are written in Latin script using almost the
same set of characters although Finnish orthography uses three
characters usually not found in English text å, ä and ö. Moreover,
there are similarities in the lexical inventories of the languages
because, like all modern languages, Finnish has borrowed a lot of
words from English and because both languages are historically
associated with Germanic and Nordic languages. In some respects,
however, Finnish and English are vastly different. Whereas English has
fixed SVO word order, the word order in Finnish is quite
flexible. Another major difference is the amount of inflectional
morphology. For example, English nouns usually only occur in singular
and plural form and may take a possessive suffix. In contrast,
thousands of inflected forms can be coined from a single Finnish noun.

Although data driven methods have dominated the field of POS tagging
and, to a lesser extent, morphological tagging for the last twenty
years, data driven work on Finnish morphological tagging has been
scarce mostly because the lack of high quality manually annotated
broad coverage training corpora. However, other approaches like the
purely rule based constraint grammar \citep{Karlsson1995} and its
derivative functional dependency grammar \citep{Tapanainen1997} have
been successfully applied for joint morphological tagging and
shallow parsing.\footnote{For example, the Finnish constraint grammar
  tagger FinCG is available online through the GiellaTekno Project
  \citep{gt}
  \url{https://victorio.uit.no/langtech/trunk/kt/fin/src/fin-dis.cg1}
  (fetched on February 24 2016).}

The recently published FinnTreeBank \citep{Voutilainen2011} and Turku
Dependence Treebank \citep{Haverinen2013} represent the first freely
available broad coverage Finnish hand labeled morphologically tagged
data sets. Thus it is now possible to conduct experiments on
morphological tagging for Finnish using a convincing gold standard
corpus. Moreover, the broad coverage open-source Finnish morphological
analyzer OMorFi \citep{Pirinen2011} is a valuable resource for
improving the performance of a tagging system. 

The rich morphology present in the Finnish language leads to problems
when existing tagging algorithms are used. The shear amount of
possible morphological analyses for a word slows down both model
estimation and application of the tagger on input text. Moreover, the
large amount of possible analyses causes data sparsity
problems. Another problem is caused by productive compounding and
extensive inflection.  Data driven methods typically perform much
worse on so called out-of-vocabulary (OOV) words, that is words which
are missing from the training corpus. In English, this is usually not
detrimental to the performance of the tagger, when input data comes
from the same domain as the training data, because the amount of OOV
words is typically rather low. In contrast, this becomes a substantial
problem for purely data driven systems processing morphologically rich
languages because productive compounding and extensive inflection lead
to a large amount of OOV words even within the same domain.

%\begin{itemize}
%\item Data-driven methods have been very successful in dealing with
%  English language technology.
%\item Morphologically complex language present problems for many statistical approaches.
%\item Data sparsity because of large vocabularies.
%\item Data sparsity because of large label sets.
%\item Model estimation is slow.
%\item Data sets are typically small.
%\item Most work on Finnish morphological tagging and disambiguation
%  has centered on purely expert driven systems such as Constraint
%  grammar.
%\item Wanted to investigate statistical methods.
%\item OMorFi, the open morphological analyzer and FTB and TDT allow
%  for work on statistical tagging of Finnish..
%\end{itemize}

\section{Main Contributions}

This thesis presents an investigation into data-driven morphological
tagging of Finnish both using generative and discriminative
models. The aim of my work has been creation of practicable taggers
for morphologically rich languages. Therefore, the main contributions
of this thesis are practical in nature. I present methods for
improving tagging accuracy, estimation speed, tagging speed and reduce
model size. More specifically, the main contributions of the thesis
are as follows.

\begin{itemize}
\item {\bf A novel formulation of generative morphological taggers
    using weighted finite-state transducers (WFST)} Finite-state
  calculus allows for flexible model specification while still
  guaranteeing efficient application of the taggers. Traditional
  generative taggers which are based on the Hidden Markov Model (HMM)
  employ a very limited feature set and changes to this feature set
  require modifications to the core algorithms of the taggers. Using
  WFSTs a more flexible feature set can, however, be employed without
  any changes to the core algorithms. This work is presented in
  Publications \ref{pub:1} and \ref{pub:2}.
\item {\bf Morphological taggers and POS taggers are applied to context
  sensitive spelling correction} Typically, context sensitive spelling
  correctors rely on neighboring words when estimating the probability
  of correction candidates. For morphologically rich languages, this
  approach fails because of data sparsity. Instead, a generative
  morphological tagger can be used score suggestions based on
  syntactic context as shown in Publication \ref{pub:3}.
\item {\bf Feature extraction specifically aimed at morphologically
    rich languages} As mentioned above, the large inventory of
  morphological labels causes data sparsity problems for morphological
  tagging models such as the averaged perceptron and conditional
  random field. Using sub label dependencies presented in Publication \ref{pub:5},
  data sparsity can, however, be alleviated. Moreover, sub-label
  dependencies allow for modeling congruence and other similar
  syntactic phenomena.
\item {\bf Faster estimation for perceptron taggers} Exact estimation
  and inference is infeasible in discriminative taggers for
  morphologically rich languages because the time requirement of exact
  estimation and inference algorithms depends on the size of the
  morphological label inventory which can be quite large. Some design
  choices (like higher model order) can even be impossible for
  morphologically rich languages using stadard tagging
  techniques. Although the speed of tagging systems is not always seen
  as a major concern, I believe that both estimation and tagging speed
  is important. A faster and less accurate tagger can often be
  preferable compared to more accurate but slower tagger in real world
  applications where high throughput is vital. Estimation speed, in
  turn, because it affects the development process of the tagger. For
  these reasons, Publications \ref{pub:4} and \ref{pub:5}
  explore known and novel approximate inference and estimation
  techniques. I show that these lead to substantial reduction in
  training times and faster tagging times compared to available
  state-of-the-art tagging toolkits.
\item {\bf Pruning strategies for perceptron taggers} Model size can
  be a factor in some applications For example, on mobile devices. In
  Chapter \ref{chapter:crf} I review different techniques for feature
  pruning for perceptron taggers and present some experiments on
  feature pruning in Chapter \ref{chapter:finnpos}.
\item {\bf FinnPos toolkit.} Publication \ref{pub:6} presents FinnPos,
  an efficient open source morphological tagging toolkit for Finnish
  and other morphologically rich languages. Chapter
  \ref{chapter:finnpos} presents a number of experiments on
  morphological tagging of Finnish using the FTB and TDT
  corpora. These experiments augment the results presented in
  Publication \ref{pub:6}.
\end{itemize}

%\begin{itemize}
%\item Investigation of statistical morphological tagging for Finnish.
%\item A formulation of generative taggers using finite-state machines
%  which allows for generalizations of the usual HMM tagger
%  formulation.
%\item Investigation of using a POS tagger to improve spelling
%  correction. Comparison against word n-grams show that POS tags
%  perform better.
%\item A new way to specify structued features.
%\item Faster estimation compared to other state-of-the-art toolkits.
%\item Investigation into averaged perceptron model pruning.
%\item FinnPos toolkit.
%\end{itemize}

\section{Outline}
This thesis can be seen as an introduction to the field of
morphological tagging and the techniques used in the field. It should
give sufficient background information for reading the articles that
accompany the thesis. Additionally, the thesis presents detailed
experiments using the FinnPos morphological tagger that were not
included in Publication \ref{pub:6}.

Chapter \ref{chap:morphology} establishes the terminology on
morphology and morphological tagging as well as surveys the field of
morphological tagging.  Chapter \ref{chap:ml} is a brief introduction
to supervised machine learning and the experimental methodology of
natural language processing.  In Chapter \ref{chapter:hmm}, I
introduce generative data-driven models for morphological tagging.
Chapter \ref{chap:fsm} introduces finite-state machines and a
formulation of generative morphological taggers in finite-state
algebra. It also shows how finite-state algebra can be used to
formulate generative taggers in a generalized manner encompassing
traditional HMM taggers and but also other kinds of models.  Chapter
\ref{chapter:crf} deals with discriminative morphological taggers and
introduces the contributions of the author to the field of
discriminative morphological tagging.  Chapter
\ref{chapter:lemmatization} deals with the topic of data-driven
lemmatization. Experiments on morphological tagging using the FinnPos
toolkit are presented in Chapter \ref{chapter:finnpos}. Finally, the
thesis is concluded in Chapter \ref{chapter:conclusions}.
