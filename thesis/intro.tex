\chapter{Introduction}
\label{ch:intro}

A morphological tagger is a piece of computer software that provides
complete morphological descriptions of sentences. An example is given
in Figure \ref{fig:mt-example}. Each of the words in the sentence is
assigned a detailed morphological label, which specifies
part-of-speech and inflectional information. Each word also receives a
lemma. Morphological tagging is typically a pre-processing step for
other language processing applications, for example syntactic parsers,
machine translation software and named entity recognizers.

The task of morphological tagging is closely related to part-of-speech
tagging (POS tagging), where the words in a sentence are tagged using
coarse morphological labels such as noun and verb. These typically
correspond to main word classes. POS taggers are sufficient for
processing languages where the scope of productive morphology is
restricted, for example English. Morphological taggers are, however,
necessary when processing {\it morphologically complex languages}, which
extensively utilize inflection, derivation and compounding for
encoding structural and semantic information. For these languages, a
coarse POS tag simply does not provide enough information to enable
accurate downstream processing such as syntactic parsing.\footnote{For
  example in Finnish, the subject and object of a sentence are
  distinguished by case and different verbs can require different
  cases for their arguments \cite{Hakulinen2004}. Coarse POS tags do
  not capture such distinctions. Therefore, accurate parsing of
  Finnish cannot rely solely on coarse POS tags.}

\begin{figure}[!htb]
\begin{center}
\begin{tabular}{|l|l|l|l|l|l|l|}
\hline
{\small\texttt{Article+Indef}} & {\small\texttt{Noun+Sg}} & {\small\texttt{Verb+Pres+3sg}} & {\small\texttt{Prep}} & {\small\texttt{Article+Def}} & {\small\texttt{Noun+Sg}} & .\\  
{\small\uppercase{a}} & {\small\uppercase{dog}} & {\small\uppercase{sleep}} & {\small\uppercase{on}} & {\small\uppercase{the}} & {\small\uppercase{mat}} & {\small\uppercase{.}}\\
\hline
A & dog & sleeps & on & the & mat & .\\
\hline
\end{tabular}
\end{center}
\caption{A morphologically tagged sentence}\label{fig:mt-example}
\end{figure}

At first glance, the task of assigning morphological descriptions, or
{\it morphological labels}, seems almost trivial. Simply form a list of
common word forms and their morphological labels and look up words in
the list when tagging text. Unfortunately, this approach
fails because of the following reasons.

\begin{enumerate}
\item A single word form can get several morphological labels
  depending on context. For example ``dog'' and ``man'' can be both
  nouns and verbs in English.
\item For morphologically complex languages, it is impossible to form a
  list of common word forms which would have sufficient coverage (say,
  higher than 95\%) on unseen text.
\end{enumerate}

Due to the reasons mentioned above, a highly accurate morphological
tagger must model the context of words in order to be able to
disambiguate between their alternative analyses. Moreover, it has to
model the internal structure of words in order to be able to assign
morphological labels for previously unseen word forms based on similar
known words.

This thesis presents work on building morphological taggers for
morphologically complex languages, in particular Finnish, which is the
native language of the author. The thesis focuses on data-driven
methods which utilize manually prepared training corpora and machine
learning techniques for learning tagger models.

\section{Motivation}
Data-driven methods have dominated the field of natural language
processing (NLP) since the 1990's. Although these methods have been
applied to virtually all language processing tasks, research has
predominantly focused on a few languages, English in particular. For
many languages with fewer speakers, such as Finnish, statistical
methods have not been applied to the same extent. This is probably due
to the fact that large training corpora required by supervised
data-driven methods are available for very few languages.

The relative lack of work on statistical NLP for languages besides English is a
problem because the languages of the world differ substantially with
regard to syntax, morphology, phonology and orthography. These
differences have very real consequences for the design of NLP
systems. Therefore, it is impossible to make general claims about
language processing without testing the claims on other languages
in addition to English.

This thesis presents work that focuses on data-driven methods for
morphological tagging of Finnish. Finnish and English share many
characteristics but also differ in many respects. Both are written in
Latin script using similar character inventories, although Finnish
orthography uses three characters usually not found in English text
``å'', ``ä'' and ``ö''. Moreover, there are similarities in the
lexical inventories of the languages because, like many modern
languages, Finnish has been influenced by English and because both
languages are historically associated with Germanic and Nordic
languages. In some respects, however, Finnish and English are vastly
different. Whereas English has fixed SVO word order, the word order in
Finnish is quite flexible. Another major difference is the amount of
inflectional morphology. For example, English nouns usually only occur
in three inflected forms: singular ``cat'', plural ``cats'' and
possessed ``cat's''. In contrast, thousands of inflected forms can be
coined from a single Finnish noun.

Although data driven methods have dominated the field of POS tagging
and, to a lesser extent, morphological tagging for the last twenty
years, data driven work on Finnish morphological tagging has been
scarce mostly because of the lack of high quality manually annotated
broad coverage training corpora. However, other approaches like the
purely rule based constraint grammar \citep{Karlsson1995} and its
derivative functional dependency grammar \citep{Tapanainen1997} have
been successfully applied for joint morphological tagging and
shallow parsing.\footnote{For example, the Finnish constraint grammar
  tagger FinCG is available online through the GiellaTekno Project\\
  %\citep{gt}
  \url{https://victorio.uit.no/langtech/trunk/kt/fin/src/fin-dis.cg1}
  (fetched on February 24, 2016).}

The recently published FinnTreeBank \citep{Voutilainen2011} and Turku
Dependence Treebank \citep{Haverinen2013} represent the first freely
available broad coverage Finnish hand labeled data sets that can be
used for work on morphological tagging. These resources enable
experiments on statistical morphological tagging for Finnish using a
convincing gold standard corpus. Moreover, the broad coverage
open-source Finnish morphological analyzer OMorFi \citep{Pirinen2011}
is a valuable resource for improving the performance of a tagging
system.

The complex morphology present in the Finnish language leads to problems
when existing tagging algorithms are used. The shear amount of
possible morphological analyses for a word slows down both model
estimation and application of the tagger on input text. Moreover, the
large amount of possible analyses causes data sparsity
problems. %Another problem is caused by productive compounding and
%extensive inflection.  

Data driven methods typically perform much
worse on so called out-of-vocabulary (OOV) words, that is words which
are missing from the training corpus, than on words seen during
training. In English, this is usually not detrimental to the
performance of the tagger, because the amount of OOV words is
typically rather low. In contrast, this becomes a substantial problem
for purely data driven systems processing morphologically complex
languages because productive compounding and extensive inflection lead
to a large amount of OOV words even within the same domain.

%\begin{itemize}
%\item Data-driven methods have been very successful in dealing with
%  English language technology.
%\item Morphologically complex language present problems for many statistical approaches.
%\item Data sparsity because of large vocabularies.
%\item Data sparsity because of large label sets.
%\item Model estimation is slow.
%\item Data sets are typically small.
%\item Most work on Finnish morphological tagging and disambiguation
%  has centered on purely expert driven systems such as Constraint
%  grammar.
%\item Wanted to investigate statistical methods.
%\item OMorFi, the open morphological analyzer and FTB and TDT allow
%  for work on statistical tagging of Finnish..
%\end{itemize}

\section{Main Contributions}

This thesis presents an investigation into data-driven morphological
tagging of Finnish both using generative and discriminative
models. The aim of my work has been creation of practicable taggers
for morphologically complex languages. Therefore, the main contributions
of this thesis are practical in nature. I present methods for
improving tagging accuracy, estimation speed, tagging speed and reducing
model size. More specifically, the main contributions of the thesis
are as follows.

\begin{itemize}
\item {\bf A novel formulation of generative morphological taggers
    using weighted finite-state machines} Finite-state calculus allows
  for flexible model specification while still guaranteeing efficient
  application of the taggers. Traditional generative taggers, which
  are based on the Hidden Markov Model (HMM), employ a very limited
  feature set and changes to this feature set require modifications to
  the core algorithms of the taggers. Using weighted finite-state
  machines, a more flexible feature set can, however, be employed
  without any changes to the core algorithms. This work is presented
  in Publications \ref{pub:1} and \ref{pub:2}.
\item {\bf Morphological taggers and POS taggers are applied to context
  sensitive spelling correction} Typically, context sensitive spelling
  correctors rely on neighboring words when estimating the probability
  of correction candidates. For morphologically complex languages, this
  approach fails because of data sparsity. Instead, a generative
  morphological tagger can be used score suggestions based on
  morphological context as shown in Publication \ref{pub:3}.
\item {\bf Feature extraction specifically aimed at morphologically
    complex languages} As mentioned above, the large inventory of
  morphological labels causes data sparsity problems for morphological
  tagging models such as the averaged perceptron and conditional
  random field. Using sub-label dependencies presented in Publication \ref{pub:5},
  data sparsity can, however, be alleviated. Moreover, sub-label
  dependencies allow for modeling congruence and other similar
  syntactic phenomena.
\item {\bf Faster estimation for perceptron taggers} Exact estimation
  and inference is infeasible in discriminative taggers for
  morphologically complex languages because the time requirement of exact
  estimation and inference algorithms depends on the size of the
  morphological label inventory which can be quite large. Some design
  choices (like higher model order) can even be impossible for
  morphologically complex languages using standard tagging
  techniques. Although the speed of tagging systems is not always seen
  as a major concern, %I believe that both estimation and tagging speed
  %is important. 
  it can be important in practice.
A faster and less accurate tagger can often be
  preferable compared to more accurate but slower taggers in real world
  applications where high throughput is vital. Estimation speed, in
  turn, is important because it affects the development process of the
  tagger. For these reasons, Publications \ref{pub:4} and \ref{pub:5}
  explore known and novel approximate inference and estimation
  techniques. It is shown that these lead to substantial reduction in
  training time and faster tagging time compared to available
  state-of-the-art tagging toolkits.
\item {\bf Pruning strategies for perceptron taggers} Model size can
  be a factor in some applications. For example, when using a tagger
  on a mobile device. In Chapter \ref{chapter:crf}, I review different
  techniques for feature pruning for perceptron taggers and present
  some experiments on feature pruning in Chapter
  \ref{chapter:finnpos}.
\item {\bf FinnPos toolkit.} Publication \ref{pub:6} presents FinnPos,
  an efficient open source morphological tagging toolkit for Finnish
  and other morphologically complex languages. Chapter
  \ref{chapter:finnpos} presents a number of experiments on
  morphological tagging of Finnish using the FTB and TDT
  corpora. These experiments augment the results presented in
  Publication \ref{pub:6}.
\end{itemize}

%\begin{itemize}
%\item Investigation of statistical morphological tagging for Finnish.
%\item A formulation of generative taggers using finite-state machines
%  which allows for generalizations of the usual HMM tagger
%  formulation.
%\item Investigation of using a POS tagger to improve spelling
%  correction. Comparison against word n-grams show that POS tags
%  perform better.
%\item A new way to specify structued features.
%\item Faster estimation compared to other state-of-the-art toolkits.
%\item Investigation into averaged perceptron model pruning.
%\item FinnPos toolkit.
%\end{itemize}

\section{Outline}
This thesis can be seen as an introduction to the field of
morphological tagging and the techniques used in the field. It should
give sufficient background information for reading the articles that
accompany the thesis. Additionally, Chapter \ref{chapter:finnpos} of
the thesis presents detailed experiments using the FinnPos
morphological tagger that were not included in Publication
\ref{pub:6}.

Chapter \ref{chap:morphology} establishes the terminology on
morphology and morphological tagging as well as surveys the field of
morphological tagging.  Chapter \ref{chap:ml} is a brief introduction
to supervised machine learning and the experimental methodology of
natural language processing.  In Chapter \ref{chapter:hmm}, I
introduce generative data-driven models for morphological tagging.
Chapter \ref{chap:fsm} introduces finite-state machines and a
formulation of generative morphological taggers in finite-state
algebra. It also shows how finite-state algebra can be used to
formulate generative taggers in a generalized manner encompassing
both traditional HMM taggers and other kinds of models.  Chapter
\ref{chapter:crf} deals with discriminative morphological taggers and
introduces contributions to the field of
discriminative morphological tagging.  Chapter
\ref{chapter:lemmatization} deals with the topic of data-driven
lemmatization. Experiments on morphological tagging using the FinnPos
toolkit are presented in Chapter \ref{chapter:finnpos}. Finally, the
thesis is concluded in Chapter \ref{chapter:conclusions}.
