\chapter{Conditional Random Fields}

\section{Discriminative modeling}

As seen in Chapter \ref{chapter:hmm}, the HMM POS tagger can be viewed
as a state machine which alternates between sampling observations,
that is words, from a state specific observation distributions, and
sampling successor states, that is the morphological labels, from
state specific transition distributions. Each emission and each
successor state is conditioned {\it solely} on the current
state. These independence assumptions are harsh. For example the
Finnish  cannot be adequately modeled, because the model does not
include direct information about neighboring words in a sentence.

Although information about for example word collocations and
orthography is quite useful in morphological labeling, it is often
difficult to incorporate more elaborate contextual information in a
generative model. As \cite{Sutton2012} note, there are two principal
methods of doing this are improving the emissions model, which may be
intractable because the model needs to account for complex
dependencies between words in a sentence and their orthographic
features, and replacing the usual emission model with for example a
Naive Bayes' model.

In a domain closely related to morphological labeling, namely
biomedical entity extraction,\cite{Ruokolainen2013} show that the
Naive Bayes approach fails. In fact, adding richer context modeling
such as adjacent words, worsens the performance of the modeling. One
reason for this may be that overlapping information sources tend to
cause the Naive Bayes model to give overly confident probability
estimates \citep{Sutton2012}. Combining them in a structured manner
can therefore be problematic.

In contrast to generative sequence models, discriminative sequence
models such as Maximum Entropy Markov Models \citep{Ratnaparkhi1998}
and Conditional Random Fields \citep{Lafferty2001} can incorporate
overlapping sources of information. They model the conditional
distribution of label sequences $p(y\cond x)$ directly instead of
modeling the joint distribution $p(x,\ y)$. Therefore, they do not
need to model the distribution of sentences at all. 

Discriminative models assign probabilities $p(y\cond x)$ for label
sequences $y$ and word sequences $x$ by extracting local features from
the input sentence and label sequence. Examples of local features
include {\it $x_t$ is ``dog'' and $y_t$ is ``NN''} and {\it $y_{t-1}$ is ``DT'' and
  $y_t$ is ``NN''}. Each feature is associated with a parameter value and the
parameter values are combined to give the conditional likelihood of
the entire label sequence. Naturally, the label sequence which
maximizes the conditional likelihood given sentence $x$ is the label
sequence returned by the discriminative POS tagger.

In generative models, emissions and transitions are independent. Both
are determined exclusively based on the current label. Contrastingly,
in discriminative models, there are no emissions or
transitions. Instead, it is customary to speak about unstructured
features relating exclusively to the input sentence, and structured
features, which incorporate information about the label
sequence. Simplifying a bit, discriminative models make no
independence assumptions among features relating to a single position
in the sentence. This allows for improved fit to training data but
parameter estimation becomes more complex as we shall soon
see. Moreover, discriminative models are more prone to
over-fitting. This is of course the famous bias-variance trade-off
\citep{Geman1992}.


\section{Maximum Entropy Modeling}
\label{sec:me}

\subsection{Example}
\label{sec:maxent-ex}

\section{Basics}
\label{crf:basics}

I will now describe a CRF POS tagger from a practical
point-of-view. The tagging procedure encompasses two stages: feature
extraction and inference using an exact or approximate inference
algorithm. Whereas inference in CRFs is very similar to inference in
HMMs, we did not discuss feature extraction in association to
HMMs. This is because HMM taggers use a restricted set of features
(the current word and preceding labels).

\begin{tabular}{lcccc}
$y$ & DT & NN & VBZ & .\\
$x$ & The & dog & eats & .
\end{tabular}

Features are true logical propositions, which connect aspects of the
input sentence with labels or parts of the label sequence. Given
sentence $x$ and label sequence $y$ of length $n$, we extract features
at every position $t$ in the sentence. For example at position $2$ in
sentence $x$, we could extract {\it The current word $x_t$ is ``dog''
  and the label is ``NN''} and {\it The previous label is ``DT'' and
  the current label is ``NN''}. We could, however, not extract the
same feature {\it The current word is ``dog'' and the label is ``NN''}
at position $1$, because this proposition is false when $t = 1$ (the
word at position 1 is ``The'' and the label is ``VBZ'').

Features are conjunctions of two parts: a feature template, for
example {\it The current word is ``dog''} and a label {\it the
  label is ``NN''}. The set of features recognized by a CRF POS tagger
contains all conjunctions $f \& y$ of feature templates $f$
present in the training data and labels $y$. For example, the tagger
would know {\it The current word is ``dog'' and the label is ``DT''}
although it is unlikely that this feature would ever be observed in
actual training data.

\cite{Ratnaparkhi1996} introduced a rather rudimentary feature set and
variations of this feature set are commonly used in the literature
(for example \cite{Collins2002} and \cite{Lafferty2001}). Let $W$ be
the set of word forms in the training data. Additionally let $P$ and
$S$ be the sets of prefixes and suffixes of maximal length $4$ of all
words $w \in W$. Then, the Ratnaparkhi feature set contains the
unstructured feature templates in Table \ref{tab:uratna} and the
structured feature templates in Table \ref{tab:sratna}.

\begin{table}[!htb]
\begin{tabular}{ll}  
Feature template & Example\\
\hline
The current word is $w$ & The current word is ``dog''\\
The current word has prefix $p$ & The current word has prefix ``d-''\\
The current word has suffix $s$ & The current word has suffix ``-og''\\
The current word contains a digit & \\
The current word contains a hyphen & \\
The current word contains a upper case letter & \\
The previous word is $w$ & The previous word is ``The''. \\
The next word is $w$ & The next word is ``eats''. \\
The word before the previous word is $w$ & \\
The word after the next word is $w$ & \\
\end{tabular}
\caption{foo}\label{tab:uratna}
\end{table}

\begin{table}[!htb]
\begin{tabular}{ll}
Feature template & Example\\
\hline
The label of the previous word is $y$ & The label of the previous word is ``NN''. \\
The label of the previous two words are $y'$ and $y$ & The labels of the two previous words are\\
 & ``DT'' and ``NN''. 
\end{tabular}
\caption{foo}\label{tab:sratna}
\end{table}

These feature templates are then combined with all labels occurring in
the training set. It is instructive to try to estimate the number of
features when using a realistic training set of size around a million
words. The number of features is be $O(\max(|Y|^2,\ |W|)\cdot
|Y|)$. For small label sets and large training data, the bulk of the
feature set tends to consist of unstructured features. However, for
large label sets in the order of $1,000$ labels, there will be a
significant number of structured features (one billion in this
case). This necessitates either dropping second order structured
features or using sparse feature representations. All structured
features simply cannot be represented in memory. We will see
techniques to circumvent these problems. Especially the averaged
perceptron is essential.

It is common to represent the CRF using linear algebra. Each position
$t$ in the sentence is represented as a vector $\phi_t$ whose
dimensions correspond to the entire space of possible features. The
selection of features is finite because it is limited by the training
data. There are only finitely many word forms, suffixes, morphological
labels and so on in the training data. The elements of each vector
$\phi_t$ represent activations of features. In the present work all
elements are either $0$ or $1$ mirroring false and true truth values,
but other activations in $\mathbb{R}$ can also be used, if the truth
values of the feature propositions exist on a non-binary scale.

In order to represent sentence positions as vectors, we need an
injective index function $I$ which maps features onto consecutive
integers starting at 1. For each feature $f$, $I(f)$ will be an
element in $\phi_t$. In a concrete implementations, the index function
$I$ could be implemented as a hash function.
 
Given a sentence $x$ and label sequence $y$, we can extract the set of features
$F_t(x)$ for each position $t$ in $x$. Let $\phi_t \in \mathbb{R}^N$ be a vector defined by
$$\phi_t(i) = 1,\ {\rm\ if}\ i \le N\ {\rm and}\ I(f) = i\ {\rm for\ some\ }f\in F_t$$
all other entries in $\phi_t$ are $0$. 

Given a parameter vector $\omega \in \mathbb{R}^N$, the probability $p(y|x)$ is
$$p(y|x) \propto \prod_{t = 1}^T \exp(\omega^\top \phi_t)$$
Specifically, the same parameter vector $\omega$ is shared by all
sentence positions and the probability $p(y|x)$ is a log linear
combination of parameter values in $\omega$.

\cite{Lafferty2001}
\section{Logistic Regression}
%\begin{itemize}
%\item Detailed view of logistic regression. 
%\item Relation to maximum entropy.
%\item Why are MEMMs not better than HMMs?
%\item Label and observation bias.
%\end{itemize}

The simplest Conditional Random Field is the {\it Logistic Regression
  Model} (LRM). It is an unstructured probabilistic discriminative
model. In this section, I will present a formal treatment of the
LRM because it aids in understanding more general
CRFs.

\subsection{The Model}
\label{sec:lr-model}

Regular linear regression models a {\it real valued quantity} $y$
based on independent variables $x_1$, ..., $x_n$. In contrast the LRM
is a regression model which models {\it the probability} that an
observation $x$ belongs to a class $y$ in a finite class set $Y$. For
example, the logistic classifier can be used to model the probability
of a tumor belonging to the class {\sc MALIGNANT} or BENIGN. The
probability is based on quantifiable information about the tumor such
as its size, shape and the degrees of expression of different genes in
the tumor cells. These quantifiable information sources are the {\it
  feature templates} of the logistic classifier and combined with
class labels they make up the features of the
model. %In concrete terms, let $f$ be the feature template {\it Tumor diameter greater then 5 cm} and $y$ be the class MALIGNANT, then we can form a feature $f \& y$ ``{\it Tumor diameter greater than 5 cm} and the tumor is MALIGNANT''.

The material at hand deals with linguistic data where most information
sources are binary, for example whether a word ends in suffix ``-s''
and whether a word is preceded by the word ``an''. In other domains
such as medical diagnostics, more general features can be used. These
can be real valued numerical measurements such as the diameter of a
tumor. This treatment of logistic classifiers will focus on the binary
valued case. When using binary features, we can equate the example $x$
with the set of feature templates $F_x \subset F$ that it {\it
  activates}, that is {\it Tumor diameter $\geq$ 5 cm}, {\it Preceded
  by ``an''} and so on. Examples that activate the exactly same
feature templates will be indistinguishable from the point of view of the
Logistic Regression model.

The logistic classifier associates each combination of a feature
template and class with a unique feature and a corresponding real
valued parameter. Intuitively, the logistic classifier models
correlations of feature templates and classes by changing the
parameter values of the associated features. For example, it might
associate the feature template {\it Tumor diameter $\geq$ 5 cm} more
strongly with the class MALIGNANT than the class BENIGN if large
tumors are cancerous more often than smaller ones. This could be
reflected in the parameter values of the model that correspond to the
features $f =$ {\it Tumor diameter $\geq$ 5 cm and class is MALIGNANT}
and $f' =$ {\it Tumor diameter $\geq$ 5 cm and class is BENIGN} so
that the parameter value for $f$ is greater than the parameter value
for $f'$. In general parameter values, however, also depend on other
features and feature correlations in the model. Therefore we can say
that the parameter value of, $f$ will be guaranteed to be greater than
the parameter value of $f'$ when $f$ is the sole feature template and
the model accurately reflects the original distribution of class
labels among examples. In the general case, where there are several
feature templates, this might fail to hold.

Formalizing the notation used in Section \ref{crf:basics}, let $F$ be
a finite set of feature templates and $Y$ a finite set of
classes. Each combination of feature template $f \in F$ and class $y
\in Y$ corresponds to a unique feature. Therefore, the model will have
$|F \times Y|$ features in total. Let $\theta$ be a real valued
parameter vector in $\R^{|F \times Y|}$ and let ${\mathrm I}$ be a
1-to-1 index function which maps each combination of feature template
and class onto the indexes of $\theta$, that is $1 \leq {\mathrm I}(f,
y) \leq |F \times Y|$.

For each example $x$, let $F_x$ be the set of feature templates that $x$ activates and let $y \in Y$ be a class. Then the feature vector associated with $x$ and $y$ is $\phi(x,y) = \{0, 1\}^{|F \times Y|}$ defined by
\[
  \phi(x,y)[i] = \left\{
  \begin{array}{ll}
  1 & {\rm iff\ } i = {\mathrm I}(f,y) {\rm\ for\ some\ }f \in F_x{\rm,} \\ 
  0 & {\rm otherwise.}  
  \end{array}
  \right.
\]

%$$F_y(I(f, y)) = 1\ {\rm iff\ }f \& y\ {\rm is\ true}.$$

Now the conditional probability $p(y\cond x)$ defining the Logistic
classifier is given by Equation \eqref{eq:logistic}. The equation
defines a probability distribution over the set of classes $Y$ because
each quantity $p(y\cond x\parcond\theta)$ is a positive real and the
quantities sum to 1. 

\begin{equation}
p(y\cond x\parcond \theta) = \frac{\exp(\theta^\top \phi(x,y))}{\sum_{z \in Y}\exp(\theta^\top \phi(x,z))}\label{eq:logistic}
\end{equation}

Often a special bias term is used in Equation \ref{eq:logistic}. It is 

The Logistic Regression Model is log-linear as demonstrated by
Equation \ref{eq:logistic,log}, which represents the model using a
cost function.

\begin{equation}
-\log p(y\cond x\parcond \theta) = log(\Z(x;\theta)) - \theta^\top F_y(x),\ {\rm where}\ Z(x;\theta) = \sum_{z \in Y}\exp(\theta^\top F_{z}(x))\label{eq:logistic,log}
\end{equation}


Given labeled training data $\data = \{(x_1,\ y_1),\ ...,\
(x_n,\ y_n)\}$, there exist several options for estimating the
parameters $\theta$. The most commonly used is the maximum likelihood,
or equivalently minimum cost, estimation. The minimum cost estimate
for the parameters $\theta$ using $\data$ is given by equation
\eqref{eq:logistic,ml}, where $\Z(\mathcal{D};\theta) = \sum_{(x,\ y) \in
  \mathcal{D}} \Z(x;\theta)$ is the partition function of the entire
data set $\data$.

\begin{equation}
\theta = \argmin_{\theta'}\Big( \log(\Z(\mathcal{D};\theta)) - \sum_{(x,\ y) \in \mathcal{D}} {\theta'}^\top F_y(x)\Big)\label{eq:logistic,ml}
\end{equation}

The probability $p(y\cond x\parcond \theta)$ has exponential form,
which means that the probability is proportional to a product of
factors of the form $e^{ap}$, where $a$ is an activation (0 or 1) and
$p$ is a parameter. This has three significant consequences:

\begin{enumerate}
\item The function $\theta \mapsto p(y\cond x\parcond \theta)$ is smooth,
\item it is convex,
\item there exists a {\it unique} $\theta$ maximizing the likelihood of the training data $\mathcal{D}$, and
\item the model $p(y\cond x\parcond \theta)$ is maximally unbiased. 
\end{enumerate}

Smoothness follows from the fact that each factor $a \mapsto e^{ap}$
is smooth and products and sums of smooth functions are
smooth. Convexity of the likelihood follows by a straightforward application of the Hölder inequality \cite{}. Property 3 is a consequence of properties 1 and 2 and
Property 4 follows from the discussion in Section \ref{sec:me}.

The log-linear form of the logistic classifier is motivated by the
concept of maximum entropy, which was introduced in Subsection
\ref{sec:maxent-ex}. 

Although the maximization in Equation \ref{eq:logistic} cannot be
solved for exactly in general, the convexity and smoothness of
$p(y\cond x\parcond \theta)$ mean that efficient numerical methods can
be used for approximating the maximum to an arbitrary precision.

\section{The Logistic Regression Model as a Classifier}

Inference for a Logistic Regression Model means finding the
probability of each class label $y \in Y$ given example $x$. The full
computation of the probability is, however, not needed when the model
is used as a classifier. For simply finding the class $y_{max}$ which
maximizes the conditional likelihood in equation
\ref{eq:logistic,inference} given fixed parameters $\theta$ it is
sufficient to maximize the numerator of $p(y \cond x\parcond\theta)$.

\begin{equation}y_{max} = \argmax_{y \in Y} p(y\cond x\parcond \theta) = \argmax_{y\in Y} \frac{\exp(\theta^\top \phi(x,y))}{\Z(x;\theta)} = \argmax_{y \in Y} \exp(\theta^\top \phi(x,y))\label{eq:logistic,inference}\end{equation}

To avoid underflow when using finite precision real numbers (such as
floating-point numbers), the maximization is usually rephrased as the
minimization of a cost-function in Equation
\ref{eq:logistic,inference,log}

\begin{equation}y_{max} = \argmin_{y\in Y} \theta^\top \phi(x,y)\label{eq:logistic,inference,log}\end{equation}

From a practical implementation perspective, the minimization in
Equation \ref{eq:logistic,inference,log} boils down to computing one
inner product $\theta^\top \phi(x,y)$ for each label $y \in Y$ and
finding the minimum. Using a suitable sparse approach each of the
inner products can be computed in $\O(|F_x|)$ time, where $F_x$ is the
set of feature templates activated by example $x$. Therefore, the
worst-case complexity of classification is dependent on the size of
the label set $Y$ and the number of feature templates $f \in F$, that
is the complexity is $\O(|Y||F|)$.

\section{Estimation}

As seen in Section \ref{lr-model}, the Logistic Regression Model is
smooth, that is the likelihood $p(y_1 \cond x_1 \parcond \theta)
... p(y_n \cond x_n \parcond \theta)$ of a labeled training data set
$\mathcal{D} = \{(x_1,y_1), ...,(x_n,y_n)\}$ is a smooth function of
the parameter vector $\theta$. This allows formulation of the
parameter estimation task for the Logistic Regression Model as a
optimization task whose solution can be approximated using numerical
techniques.


\section{CRF -- A Structured Logistic Classifier}

\section{Note on Terminology}
\begin{itemize}
\item The term perceptron tagger is commonly found in the litterature, e.g. \cite{Collins2002}.
\item The model is called a CRF.
\item The estimator can be a ML, Perceptron, pseudo likelihood, pseudo
  perceptron...
\end{itemize}

\section{Inference}
\begin{itemize}
\item Inference using Viterbi.
\end{itemize}

\section{Estimation}
\begin{itemize}
\item Iterative process: label data, adjust parameters and repeat.
\item Use SGD or L-BFGS \citep{Vishwanathan2006} for ML estimation. 
\item Use averaged perceptron \citep{Collins2002}.
\item Dev data for adjusting hyper parameters: number of iterations and
  regularization hyper-parameters, model order.
\item Problems with exact estimation.
\item Layered CRF \cite{Mueller2013}, \cite{Weiss2010} and \cite{Charniak2005}.
\end{itemize}

\section{Approximate Estimation}
\begin{itemize}
\item Beam search.
\item Violation fixing.
\item Label guessing.
\end{itemize}
\section{CRF taggers and Morphological Analyzers}

\section{Enriching the Structured Model}\label{sec:sub-labels}
\begin{itemize}
\item Utilizing sub-label structure.
\end{itemize}

