\chapter*{List of Publications}
\addcontentsline{toc}{chapter}{List of Publications}
\markboth{List of Publications}{}
\nobibliography*
\begin{enumerate}[label=\textbf{\Roman*}]
\item\label{pub:1}\noindent\bibentry{Silfverberg2010}\\[1cm]

\item\label{pub:2}\noindent\bibentry{Silfverberg2011}\\[1cm]

\item\label{pub:3}\noindent\bibentry{Pirinen2012}\\[1cm]

\item\label{pub:4}\noindent\bibentry{Ruokolainen2014}\\[1cm]

\item\label{pub:5}\noindent\bibentry{Silfverberg2014}\\[1cm]

\item\label{pub:6}\noindent\bibentry{Silfverberg2015}
\end{enumerate}
\chapter*{Author's Contribution}
\addcontentsline{toc}{chapter}{Author's Contributions}
\markboth{Author's Contributions}{}
\noindent{\large{\bf Publication \ref{pub:1}: Part-of-Speech Tagging Using Parallel Weighted Finite-State Transducers}}
\vskip.2cm
\noindent The paper presents a weighted finite-state
implementation for hidden Markov models. I developed
the system, performed the experiments and wrote the paper under
supervision of the second author Krister Lind\'{e}n.
\vskip1cm
\noindent{\large{\bf Publication \ref{pub:2}: Combining Statistical Models for {POS} Tagging using Finite-State Calculus}}
\vskip.2cm
\noindent The paper presents a continuation of the work in publication
\ref{pub:1}. It presents extensions to the standard hidden Markov
Model, which can be implemented using weighted
finite-state calculus. I developed the system,
performed the experiments and wrote the paper under supervision of
the second author Krister Lind\'{e}n.
\vskip1cm
\noindent{\large{\bf Publication \ref{pub:3}: Improving Finite-State Spell-Checker Suggestions with Part of Speech N-Grams}}
\vskip.2cm
\noindent The paper presents a spell-checker, which utilizes the POS
taggers presented in publications \ref{pub:1} and \ref{pub:2} for
language modeling. I performed the experiments together
with the first author Tommi Pirinen and participated in writing the
paper under supervision of the third author Krister Lind\'{e}n.
\vskip1cm
\noindent{\large{\bf Publication \ref{pub:4}: Accelerated Estimation of
  Conditional Random Fields using a Pseudo-Likelihood-inspired
  Perceptron Variant}} 
\vskip.2cm
\noindent The paper presents a variant of the perceptron
algorithm inspired by the pseudo-likelihood estimator for conditional
random fields. I performed the experiments and
participated in the writing of the paper under supervision of the
third author Krister Lind\'{e}n.
\vskip1cm
\noindent{\large{\bf Publication \ref{pub:5}: Part-of-Speech Tagging using
  Conditional Random Fields: Exploiting Sub-Label Dependencies for
  Improved Accuracy}} 
\vskip.2cm
\noindent The paper presents a system which uses
dependencies of the components of structured morphological labels
for improving tagging accuracy. I implemented the
system presented in the paper, performed the experiments and
participated in the writing of the paper under supervision of the
third author Krister Lind\'{e}n.
\vskip1cm
\noindent{\large {\bf Publication \ref{pub:6}: FinnPos: an open-source morphological tagging and lemmatization toolkit for Finnish}}
\vskip.2cm
\noindent The paper presents a morphological tagging toolkit for Finnish and
other morphologically rich languages. The present author and the
second author Teemu Ruokolainen designed the system and the
methodology of the paper together. I implemented the
FinnPos toolkit presented in the paper, performed the experiments and
participated in the writing of the paper under supervision of the
third author Krister Lind\'{e}n.
