\chapter{Conclusions and Future Work}
\label{chapter:conclusions}

This thesis has presented work on data driven morphological tagging
for Finnish using both generative and discriminative models.

\paragraph{Generative Taggers} The finite-state implementation of
generative taggers which is presented in Publications \ref{pub:1} and
\ref{pub:2} allows for flexible model formulation. Publication
\ref{pub:2} shows that it compares favorably to the widely used HunPos
tagger when tagging Finnish text. The implementation does, however,
not solve the principal problem of HMM taggers: the independence
assumptions in the model are too harsh. Therefore, complex
unstructured features such as word context cannot be used. This is
probably the greatest pitfall of generative models because surrounding
words are quite useful in morphological tagging.

There are other reasons to prefer discriminative models above the
generative tagging paradigm. All generative models require some manner
of smoothing. It is difficult to know what the optimal choice of
smoothing method. This may even be language specific to some
extent. For example, Publication \ref{pub:2} indicates that the
guesser presented in \cite{Brants2000} may not be optimally suited for
morphologically rich languages such as Finnish. A guesser based on the
longest common suffix with words in the training data may give better
results.

\paragraph{Discriminative Taggers} The discriminative model presented
in this thesis is based on the averaged perceptron model. It
incorporates sub-label dependencies to improve accuracy in presence of
large structured label sets and a cascaded model structure and beam
search in order to speed up estimation. Moreover, I investigated
different pruning strategies for models and showed that model size can
be reduced by up to 80\% with negligible reduction in accuracy. The
FinnPos toolkit implements these optimizations and is freely available
as an open-source utility.

The experiments in Chapter \ref{chapter:finnpos} show that the
morphological analyzer is clearly the most influential factor for the
accuracy of the model. It results in much larger gains in accuracy
than increasing model order or using sub-label dependencies. Sub-label
dependencies however are equally or more influential than increased
model order. This is not entirely surprising because a second order
model will be very sparse when used in presence of label sets
exceeding a thousand labels and training data on the order of 200,000
tokens. In contrast, sub label dependencies are by definition less
sparse.

\paragraph{Future Work} It would be interesting to try self-training
as presented by \cite{Sogaard2011}. Other semi-supervised training
methods such as distributional similarity could also be
interesting. In the context of morphologically rich languages,
distributional similarity of word forms might require a large amount
of training data. Therefore, it could also be interesting to explore
using lemmatized training material.

All experiments in this thesis have used a full-fledged morphological
analyzer. It would be interesting to try out a morphological
segmentation application such as Morfessor \citep{Creutz2002}. The
segments could be used as features in a similar manner as the
morphological labels are used in the current system.

Further feature engineering could probably be useful. For example verb
valency could be useful. It would also be interesting to combine the
finite-state implementation presented in Publications \ref{pub:5} and
\ref{pub:6} with the discriminative estimation in the FinnPos
toolkit. Especially, it would be interesting to explore global tagger
constraints implemented as features in a discriminative tagger. An
simple example of such a feature is {\bf the sentence has a finite
  verb form}. This would be possible using the finite-state
implementation which is not constrained by a fixed model order unlike
standard inference using the Viterbi algorithm. However, efficient
estimation would probably be a challenge.

As further work on model pruning, it would be interesting to compare
value based pruning and L1 regularization for perceptron taggers and
investigate the combination of these methods.

%\begin{itemize}
%\item Including disambiguation rules for Finnish.
%\item Further feature engineering especially in the form of
%  linguistically relevant gazzetteers.
%\item Semi-supervised approach using self-training.
%\item Other possible semi-supervised approaches.
%\item Using morfessor instead of a morphological analyzer.
%\item Using a data-driven morphological analyzer instead of a regular
%  one.
%\end{itemize}


