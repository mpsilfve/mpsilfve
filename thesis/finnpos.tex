\chapter{Experiments}
\label{chapter:finnpos}

FinnPos, presented in Publication \ref{pub:6}, is a discriminative
morphological tagger based on the averaged perceptron classifier. It
is specifically geared toward morphologically rich languages. To this
end, it implements beam search and a cascaded model architecture for
speeding up estimation and tagging speed. To improve accuracy, it
includes an option to use a morphological analyzer during
tagging. Moreover, it extracts sub-label dependencies for unstructured
and structured features in order to improve tagging accuracy in presence of
large structured label sets.

This chapter presents experiments on morphological tagging in Finnish
using FinnPos. These experiments are intended to augment the
experiments in Publication \ref{pub:6}. I will present the following
experiments

\begin{itemize}
\item {\bf Cascaded Model Architecture} What is the effect of different settings for the label guesser on tagging accuracy, estimation speed and tagging speed?
\item {\bf Beam Search} What is the effect of beam width on tagging accuracy, estimation speed and tagging speed?
\item {\bf Model Order} What is the effect of model order on tagging accuracy, estimation speed and tagging speed? Experiments are presented both when tagging with and without a morphological analyzer.
\item {\bf Sub-Label Order} What is the effect of the order of sub-label dependencies on tagging accuracy, estimation speed and tagging speed? Experiments are presented both when tagging with and without a morphological analyzer.
\item {\bf Model Pruning} Which is the better model pruning strategy: update count based or value based pruning?
\end{itemize}



\section{Data Sets}
All experiments are conducted on both the Turku Dependecy Treebank
\citep{Haverinen2013} (TDT) and FinnTreeBank \citep{Voutilainen2011}
(FTB). Two distinct data sets, containing text from different genres,
are used in order to justify general claims about the performance of
the system in Finnish morphological tagging. Table \ref{tab:data} gives a
numerical overview of both data sets.

\begin{table}[htb!]
%\begin{small}
\begin{center}
\begin{tabular}{l|l|l} 
 & TDT & FTB \\
\hline
size & 13,572 sent. (183,118 tok.) & 19,121 sent. (162,028 tok.) \\
\# labels & 2,014 & 1,399 \\
OMorFi coverage & 94.2\% & 99.0\% \\
\end{tabular}
\end{center}
%\end{small}
% \caption{A post-processed and disambiguated OMorFi analysis with Turku Dependency Treebank annotation for an exemplar sentence \emph{H\"an ei asu pieness\"a kyl\"ass\"a} adapted from \citet{haverinen2014}.}
\caption{Summary of Turku Dependency Treebank (TDT) \citep{Haverinen2013} and FinnTreeBank (FTB) \citep{Voutilainen2011}. The OMorFi coverage refers to coverage per token.}
\label{tab:data} 
\end{table}

\paragraph{FinnTreeBank}
FTB is a morphologically tagged and dependency parsed collection of
example sentences from Iso Suomen Kielioppi, a descriptive grammar of
the Finnish language \citep{Hakulinen2004}. The examples have been
harvested from newspapers, novels and blogs. Additionally, some
examples represent spoken language. 

The FTB corpus only contains isolated sentences and each sentence has
been selected to illustrate some grammatical phenomenon. Therefore, it
is safe to say that FTB does not represent a random sample of Finnish
text. It probably contains a high number of rare grammatical
phenomena. This can be seen as a weakness because results on the FTB
corpus may not carry over to other data sets. However, it also makes
the corpus interesting and challenging from the point of view of
morphological tagging because the data is expected to be complex.

Both the morphological tagging and dependency structures of FTB have
been manually prepared.  The morphological analyses of word tokens are
post-processed outputs of OMorFi, an open-source morphological
analyzer for Finnish
\citep{Pirinen2011}. % The post-processing steps reported by \citet[Section 5.2]{haverinen2014} comprise x, y, and z.

\paragraph{Turku Dependency Treebank} TDT contains text from ten different domains, for example Wikipedia articles, blog entries, and financial news. The annotation has been prepared by manually correcting the output of an automatic annotation process.
%The domains and the respective section sizes are presented in Table \ref{tab: TDT}. 
%In addition, the treebank contains a publicly non-available test set of 1,554 sentences (21,211 tokens) for evaluation purposes.
Similarly to FTB, the morphological analyses of word tokens in FTB are
post-processed outputs of OMorFi \citep{Pirinen2011}. However, the
treebanks are based on different versions of OMorFi. Moreover, the
post-processing steps applied in TDT and FTB differ. This results in
somewhat different annotation schemes.  The TDT annotation for each
word token consists of word lemma (base form), part-of-speech (POS),
and a list of detailed morphological information, including case,
number, tense, and so forth.

\paragraph{Data Splits} Well established data sets usually have
established splits into training, development and test data. For
example, most work on English POS tagging reports results on the Penn
Treebank corpus \citep{Marcus1993} using the data splits introduced by
\cite{Collins2002}. This is sound because it guarantees that results
reported in different papers are comparable. For the data sets used in
this thesis, Finn TreeBank \citep{Voutilainen2011} and Turku
Dependency Treebank \citep{Haverinen2013}, there are no established
splits. Therefore, the experiments use 80\% of the data for training,
10\% as development data and 10\% as final test data. The data is
split in the following way: For each consecutive ten sentences, the
first eight are assigned to the training set, the ninth one to the
development set and the tenth one to the test set.

\section{Setup for the Experiments}

%This section is intended to augment the treatment
%in Publication \ref{pub:5}. It presents several experiments on
%morphological tagging using the FinnPos tagger toolkit which were
%omitted from the paper in favor of clarity.

Because FinnPos incorporates a variety of optimizations to accuracy
and speed governed by different hyper-parameter, it is impossible to
conduct exhaustive experiments (a grid search would simply include too
many experiments). Instead I have chosen the settings presented in
Publication \ref{pub:6} as vantage point and examined the impact of
changing one hyper-parameter at a time. These settings were chosen
because they give state-of-the-art accuracy as presented in the paper.

The basic setting for all experiments is 
\begin{itemize}
\item A second order model.
\item First order sub-label dependencies.
\item 99.9\% mass for the generative label guesser.
\item 99.9\% mass for the adaptive beam search.
\end{itemize}

A morphological analyzer is only used when indicated. A label
dictionary is used in all experiments both to speed up decoding and
improve accuracy. The label dictionary is also used when the
morphological analyzer is used. The reason for this is that a liberal
compounding or derivation mechanism (such as the one implemented in
the OMorFi morphological analyzer) can result in improbable
analyses. The analyses that have been attested in the training corpus
should be preferred when possible.

The feature set used in the experiments follows Publication
\ref{pub:6}.  Let $x = (x_1 ... x_T)$ be a sentence, $y = (y_1
... y_T)$ a label sequence and $t$ and index. Then the unstructured
features templates for the morphological tagger are the familiar
Ratnaparkhi features \citep{Ratnaparkhi1998} augmented with a few
additional features. For all words, FinnPos uses the following feature
templates
\begin{itemize}
\item The word form $x_t$ and the lower cased version of $x_t$.
\item The length $|x_t|$ of word form $x_t$.
\item Each word form in a four word window around $t$: $x_{t-2}$, $x_{t-1}$, $x_{t+1}$ and $x_{t+2}$.
\item The word form bigrams $(x_{t-1}, x_t)$ and $(x_t,x_{t+1})$.
\end{itemize}
For rare words\footnote{A rare word is one that is not common. The
  list of common words is user defined but in these experiments I have
  defined common words to be words having frequency 10 or higher in
  the training corpus}, it additionally extracts the following
features
\begin{itemize}
\item {\tt DIGIT} when $x_t$ contains a digit.
\item {\tt UC} when $x_t$ contains an upper case letter.
\item Prefixes and suffixes of $x_t$ up to length 10.
\end{itemize}
When a morphological analyzer is used, each morphological label given
to word $x_t$ is also used as a feature template.

%Let $x = x_1 ... x_n$ be a word form of length $n$ and $y$ its
%label. Then the features for the lemmatizer are
%\begin{itemize}
%\item The lower cased variant of $x$.
%\item Prefixes of $x$ up to length 5 and suffixes up to length 7.
%\item The infixes $x_{n - 2}x_{n - 1}$, $x_{n - 3}x_{n - 2}$ and $x_{n - 4}x_{n - 3}$.
%\item The label $y$ and its main part-of-speech.
%\item {\tt DIGIT} when $x$ contains a digit.
%\item {\tt UC} when $x$ contains an upper case letter.
%\end{itemize}
%Additionally a combination of each feature template and the
%morphological label $y$ is used as a feature template. Naturally, the
%combination of $y$ with itself is omitted.

Some baseline runs are impossible to run: FinnPos uses a second order
model. With a label set size of 1,000, trellises used during inference
become very large and inference is prohibitively slow. Therefore, it
was not feasible to run experiments without label pruning during
training.

The results of the experiments presented here differ minutely from the
results presented in Publication \ref{pub:6} due to added features
(lower cased word form and word length) and a few bug fixes that have
improved results.

\section{Using a Cascaded Model}

\begin{table}[htb!]
\begin{center}
\begin{tabular}{cccc}
\multicolumn{4}{c}{Adaptive Guess Count}\\
Guess Mass & Tagging Accuracy (\%) & Training time & Dec. Speed (KTok/s)\\
\hline
0.9        & 93.21 (OOV: 77.68) & 3 min, 3 epochs &   7          \\
0.99       & 93.11 (OOV: 77.14) & 3 min, 3 epochs & 7            \\
0.999      & 93.23 (OOV: 78.49) & 4 min, 4 epochs            & 8            \\
0.9999      & 93.41 (OOV: 78.55) & 2 min, 2 epochs            & 7            \\
           &                  &               &               \\
\multicolumn{4}{c}{Fixed Guess Count}\\
Guess Count & Tagging Accuracy (\%) & Training time (min) & Dec. Speed (KTok/s) \\
\hline
1        & 91.48 (OOV: 69.81)           & 1 min, 3 epochs            & 8            \\
10       & 93.23 (OOV: 77.56)           & 2 min, 2 epochs            & 7            \\
20       & 93.18 (OOV: 77.89)           & 3 min, 3 epochs            & 7            \\
30       & 93.22 (OOV: 77.62)           & 4 min, 3 epochs            & 6            \\
40       & 93.43 (OOV: 78.49)           & 4 min, 2 epochs            & 5            \\
\end{tabular}
\caption{Different label guesser settings for FinnTreeBank}
\end{center}
\end{table}


\begin{table}[htb!]
\begin{center}
\begin{tabular}{cccc}
\multicolumn{4}{c}{Adaptive Guess Count}\\
Guess Mass & Tagging Accuracy (\%) & Training time (min) & Dec. Speed (KTok/s)\\
\hline
0.9        & 92.69 (OOV: 74.10)           & 8 min, 8 epochs           & 5            \\
0.99       & 92.66 (OOV: 74.13)           & 9 min, 8 epochs           & 5            \\
0.999      & 92.76 (OOV: 74.65)           & 8 min, 7 epochs            & 5            \\
0.9999      & 92.75 (OOV: 74.30)           & 6 min, 5 epochs            & 5            \\
           &                  &               &               \\
\multicolumn{4}{c}{Fixed Guess Count}\\
Guess Count & Tagging Accuracy (\%) & Training time (min) & Dec. Speed (KTok/s) \\
\hline
1        & 89.85  (OOV: 61.33)          & 2 min, 5 epochs            & 6            \\
10       & 92.35  (OOV: 72.55)          & 5 min, 7 epochs            & 6            \\
20       & 92.61  (OOV: 73.88)          & 5 min, 4 epochs            & 6            \\
30       & 92.81  (OOV: 74.87)          & 12 min, 9 epochs          & 5            \\
40       & 92.91  (OOV: 75.35)          & 14 min, 9 epochs            & 5            \\
\end{tabular}
\caption{Different label guesser settings for Turku Dependency Treebank}
\end{center}
\end{table}

This subsection presents experiments on using different settings for
the generative label guesser included as a pre-pruning step during
training as explained in Section \ref{}.

For both Turku Treebank and FinnTreeBank, I compare a fixed amount of
label guesses to choosing a varying amount of guesses per word using a
generative label guesser. It was not possible to run experiments
without any form of label pruning during training because of
prohibitive runtime and memory requirements. The most important point
of these experiments is that using some kind of label guesser during
training is almost necessary if one wants to train a second order
model for label sets of several hundreds or thousands of labels.

In general, using a larger amount of label guesses improves
accuracy. For both FTB and TDT, the accuracy levels off already at 40
guesses per word. 

For FTB, the mass 0.9999 yields approximately the same accuracy than
as 40 guesses. For TDT, however, 40 label guesses results in
$0.2\%$-points better accuracy than using the mass $0.9999$.

The training time per epoch is clearly related to the amount of label
guesses, however, the number of epochs seems to fluctuate somewhat
from two to four for FTB and from five to eleven for TDT. Therefore,
it is difficult to see a clear trend.

The amount of label guesses influences decoding speed to some degree
because the same setting is used for the label guesser during
decoding. Because the label guesser is only used for OOV words, the
exact setting of the guesser does however only have a moderate impact.

\section{Beam Search}

\begin{table}[htb!]
\begin{center}
\begin{tabular}{cccc}
\multicolumn{4}{c}{Adaptive Beam}\\
Beam Width & Tagging Accuracy (\%) & Training time (min) & Dec. Speed (KTok/s)\\
\hline
0.9        & 93.08 (OOV: 76.99)           & 3 min, 3 epochs & 6           \\
0.99       & 93.14 (OOV: 77.44)           & 3 min, 3 epochs & 8            \\
0.999      & 93.28 (OOV: 80.49)           & 2 min, 2 epochs & 8            \\
           &                  &               &               \\
\multicolumn{4}{c}{Fixed Beam}\\
Beam Width & Tagging Accuracy (\%) & Training time & Dec. Speed (KTok/s) \\
\hline
1        & 92.32 (OOV: 75.07)            & 2 min, 2 epochs & 7          \\
10       & 93.33 (OOV: 78.28)           & 2 min, 2 epochs &  7            \\
20       & 93.11 (OOV: 77.29)           & 4 min, 3 epochs &  7             \\
$\infty$ & 93.31 (OOV: 78.19)           & 20 min, 2 epochs            & 6            \\
\end{tabular}
\caption{Different beam settings for FinnTreeBank.}
\end{center}
\end{table}

\begin{table}[htb!]
\begin{center}
\begin{tabular}{cccc}
\multicolumn{4}{c}{Adaptive Beam}\\
Beam Width & Tagging Accuracy (\%) & Training time (min) & Dec. Speed (KTok/s)\\
\hline
0.9       & 92.55 (OOV: 73.73)           & 5 min, 5 epochs           & 6            \\
0.99       & 92.87 (OOV: 74.88)           & 7 min, 7 epochs            & 6            \\
0.999       & 92.76 (OOV: 74.65)            & 8 min, 7 epochs            & 6            \\
           &                  &               &               \\
\multicolumn{4}{c}{Fixed Beam}\\
Beam Width & Tagging Accuracy (\%) & Training time (min) & Dec. Speed (KTok/s) \\
\hline
1        & 91.80 (OOV: 71.26)           & 6 min, 9 epochs            & 6            \\
10       & 92.83 (OOV: 74.85)           &  7 min, 6 epochs           & 6            \\
20       & 92.60 (OOV: 73.98)           & 10 min, 7 epochs            & 6            \\
$\infty$       & 91.58??  (OOV: 69.46)          & 199 min, 8 epochs            & 6            \\
\end{tabular}
\caption{Different beam settings for Turku Dependency Treebank.}
\end{center}
\end{table}

This subsection presents experiments using different beam settings
during training and decoding. 

I compare fixed beam width to an adaptive beam presented in Section
\ref{}. Additionally, I include training and decoding results when no
beam is used.

It is difficult to see a clear relation between the beam width and
tagging accuracy. The fixed beam of width one is clearly the worst for
both TDT and FTB. However, all higher beams seem to give results in
the same range. Moreover, increasing the beam from 10 to 20 results in
a $0.1\%$-point drop in accuracy for FTB. Additionally, the system
without any beam search performs worse than systems with adaptive or
fixed beam width for TDT.

A small beam width results in a faster training time than a larger
beam. When using an infinite beam, the training time for TDT is
surprisingly large. The reason for this is that there are sequences of
words in the training and development data which receive a large
number of label guesses. When no beam is used, this results in very
long tagging times for the sequences because a second order model is
used.

Contrary to what the literature indicates
\citep{Huang2012,Collins2004}, violation fixing gave no significant
improvements in accuracy in preliminary experiments. As it only slows
down training, it was not included in FinnPos.

\section{Model Order}

\begin{table}[htb!]
\begin{center}
\begin{tabular}{cccc}
\multicolumn{4}{c}{Without a Morphological Analyzer}\\
Model Order & Tagging Accuracy (\%) & Training time & Dec. Speed (KTok/s)\\
\hline
0        & 91.91            & 3 min, 3 epochs            & 8            \\
1        & 92.49            & 3 min, 4 epochs            & 8            \\
2        & 92.49            & 4 min, 5 epochs            & 7            \\
                &                       &                  &            \\
\multicolumn{4}{c}{Using a Morphological Analyzer}\\
Model Order & Tagging Accuracy (\%) & Training time & Dec. Speed (KTok/s)\\
\hline
0        & 95.35            & 5 min, 9 epochs            & 25            \\
1        & 95.98            & 5 min, 10 epochs           & 23            \\
2        & 95.96            & 5 min, 8 epochs            & 24            \\
\end{tabular}
\caption{Different Model Orders for FinnTreeBank}
\end{center}
\end{table}

\begin{table}[htb!]
\begin{center}
\begin{tabular}{cccc}
Model Order & Tagging Accuracy (\%) & Training time (min) & Dec. Speed (KTok/s)\\
\hline
0        & 91.15            & 6 min, 4 epochs            & 6            \\
1        & 91.17            & 8 min, 8 epochs            & 5            \\
2        & 91.83            & 5 min, 3 epochs            & 5            \\
                &                       &                  &            \\
\multicolumn{4}{c}{Using a Morphological Analyzer}\\
\hline
0        & 95.53            & 5 min, 4 epochs            & 21            \\
1        & 96.05            & 5 min, 7 epochs            & 22            \\
2        & 96.13            & 5 min, 5 epochs            & 20            \\

\end{tabular}
\caption{Different Model Orders for Turku Dependency Treebank}
\end{center}
\end{table}

In this section I examine the impact of model order on tagging
accuracy, training time and decoding time. The experiments in this
section do not use sublabel features in order to clearly reveal the
impact of model the order in isolation of other factors.

The accuracy on both FTB and TDT increases when going from order zero
to a first order model. Further increasing the model order only gives
an improvement for TDT.

The increase in accuracy when going from a unstructured (order zero)
model to a second order model is approximately 0.5\%-points for both
FTB and TDT. This applies both when using a morphologial analyzer and
when not using it. It is noteworthy that the increas in accuracy
resulting from the morphological analyzer is substantially larger for
both data sets.

\section{Sub-Label Dependencies}

In this section I examine the impact of sub-label order order on
tagging accuracy, training time and decoding time both using a
morphological analyzer and without a morphological analyzer. The
results in this section differ slightly from Publication \ref{pub:6}
because of minor bug fixes and improvements to the feature set of the
tagger.

The total improvement in accuracy stemming from sub-label features is
approximately 0.8\%-points for both FTB and TDT when not using a
morphological analyzer and around 0.3\%-points when using a
morphological analyzer. Contrasting these results with the previous
section examining model order, we can see that the added improvement
from sub-label features is larger than improvement given by incresing
model from order 0 to 2 when not using the morphological
analyzer. Moreover, the imporovement is approximately approximately
the same as going from model order 0 to 1 when using a morphological
analyzer.

Only for FTB do second order sub-label features give added accuracy
compared to first order features and only when using the morphological
analyzer. In other cases, second order sub-labels perform worse than
first order sub-labels.

Training time increases and decoding speed decreases with incresing
sub-label order. However, sub-labels seems to decrease the amount of
training epochs needed to converge to the final model parameters.

For both FTB and TDT, unstructured sub-label features are more
influential for accuracy than structured sub-label features when the
morphological analyzer is not used. Then it is used, structured
sub-labels, converesly, give a larger improvement.

\begin{table}[htb!]
\begin{center}
\begin{tabular}{cccc}
\multicolumn{4}{c}{Without a Morphological Analyzer}\\
Sub-Label Order & Tagging Accuracy (\%) & Training time    & Dec. Speed (KTok/s)\\
\hline
None            & 92.49 (OOV: 74.68)    & 3 min, 5 epochs  & 6                       \\
0               & 93.05 (OOV: 77.74)    & 1 min, 2 epochs  & 5                       \\
1               & 93.29 (OOV: 78.40)    & 1 min, 2 epochs  & 4                       \\
2               & 92.68 (OOV: 75.22)    & 5 min, 4 epochs  & 6                       \\
                &                       &                  &                          \\
\multicolumn{4}{c}{Using a Morphological Analyzer}\\
Sub-Label Order & Tagging Accuracy (\%) & Training time    & Dec. Speed (KTok/s)\\
\hline
None     & 95.98 (OOV: 91.41)           & 3 min, 8 epochs  & 25                       \\
0        & 96.08 (OOV: 91.98)           & 1 min, 3 epochs  & 22                       \\
1        & 96.24 (OOV: 92.28)           & 1 min, 2 epochs  & 21                       \\
2        & 96.31 (OOV: 92.58)           & 4 min, 3 epochs  & 19                       \\
\end{tabular}
\caption{Different Sub-Label Orders for FinnTreeBank}
\end{center}
\end{table}


\begin{table}[htb!]
\begin{center}
\begin{tabular}{cccc}
\multicolumn{4}{c}{Without a Morphological Analyzer}\\
Sub-Label Order & Tagging Accuracy (\%) & Training time    & Dec. Speed (KTok/s)\\
\hline
None            & 91.89 (OOV: 70.63)    & 2 min, 3 epochs  & 5                       \\
0               &  92.59 (OOV: 73.98)   & 2 min, 4 epochs  & 5                       \\
1               &  92.69 (OOV: 74.35)   & 5 min, 7 epochs  & 3                       \\
2               &  92.31 (OOV: 72.31)   & 13 min, 8 epochs  & 5                       \\
                &                       &                  &                          \\
\multicolumn{4}{c}{Using a Morphological Analyzer}\\
Sub-Label Order & Tagging Accuracy (\%) & Training time    & Dec. Speed (KTok/s)\\
\hline
None            &  96.12 (OOV: 91.12)   & 3 min, 5 epochs  & 19                       \\
0               &  96.17 (OOV: 91.39)   & 2 min, 5 epochs  & 18                       \\
1               &  96.39 (OOV: 91.84)   & 3 min, 5 epochs  & 16                       \\
2               &  96.29 (OOV: 91.69)   & 12 min, 8 epochs  & 16                       \\
\end{tabular}
\caption{Different Sub-Label Orders for Turku Dependency Treebank}
\end{center}
\end{table}


\section{Pruning the Model}

In this section, I examine two model pruning strategies: pruning by
update count and pruning by parameter mass. The strategies are
presented in \ref{sec:pruning}.

The value for the pruning parameter was set using development
data. The range of parameter values was chosen so as to show the
difference in pruning efficiency. The specific parameter values are
not very important. The important thing is to show the relation of
model size and accuracy. In these experiments, pruning has not been
applied to the lemmatizer although that would be possible.

Clearly, mass based pruning is more effective than update count based
pruning. For FTB, without a morphological analyzer, the full accuracy
of $93.2\%$ can be maintained even when pruning out 81\% of model
parameters. When using update count as pruning criterion, full
accuracy cannot be maintained when pruning out more than 38\% of model
parameters. For TDT, the corresponding figures are 55\% for mass based
pruning and 23\% for update based pruning.

When using a morphological analyzer, even further feature pruning is
possible. For FTB, 84\% of model parameters can be pruned while
maintining full accuracy when using mass based pruning. When using
update count based pruning, however, no parameters can be pruned
without losing accuracy. For TDT, update count based pruning can prune
out 72\% of the features when using a morphological analyzer but mass
based pruning can prune out even more -- 81\%.

\begin{table}[htb!]
\begin{center}
\begin{tabular}{c|ccccc}
\multicolumn{1}{c}{}        & \multicolumn{5}{c}{Update Count Threshold}\\
MA                     & None           & $< 2$        & $< 3$        & $< 4$        & $< 5$       \\
\hline
no                     & 93.2\%, 4.8M   & 93.2\%, 3.9M & 93.2\%, 3.6M & 93.2\%, 3.0M & 93.1\%, 1.0M \\
yes                    & 96.3\%, 4.3M   & 96.2\%, 3.9M & 96.2\%, 3.7M & 96.2\%, 3.3M & 96.1\%, 1.0M \\
                       &                &              &              &              &              \\
\multicolumn{1}{c}{}                        & \multicolumn{5}{c}{Parameter Mass Threshold}\\
MA                     & None         & $< 2.0$      & $< 2.5$      & $< 3$      & $< 3.5$        \\
\hline
no                     & 93.2\%, 4.8M & 93.3\%, 1.8M & 93.2\%, 1.4M & 93.2\%, 1.2M & 93.2\%, 0.9M \\
yes                    & 96.3\%, 4.3M & 96.3\%, 0.9M & 96.3\%, 0.7M & 96.2\%, 0.3M & 96.1\%, 0.1M \\
\end{tabular}
\caption{Result of applying different pruning strategies on FinnTreeBank models.}
\end{center}
\end{table}

\begin{table}[htb!]
\begin{center}
\begin{tabular}{c|ccccc}

\end{tabular}
\caption{Result of applying different pruning strategies on Turku Dependency Treebank models.}
\end{center}
\end{table}

\begin{table}[htb!]
\begin{center}
\begin{tabular}{c|ccccc}
\multicolumn{1}{c}{}                       & \multicolumn{5}{c}{Update Count Threshold}\\
MA                     & None            & $< 2$        & $< 3$        & $< 4$        & $< 5$       \\
\hline
no                     & 92.8\%, 6.4M    & 92.7\%, 5.2M & 92.7\%, 5.0M & 92.8\%, 4.9M & 92.6\%, 4.9M \\
yes                    & 96.3\%, 5.5M    & 96.4\%, 5.0M & 96.3\%, 4.9M & 96.4\%, 4.9M & 96.3\%, 4.9M \\
                       &                 &              &              &              &              \\
\multicolumn{1}{c}{}     & \multicolumn{5}{c}{Parameter Mass Threshold}\\
MA                     & None           & $< 4.0$        & $< 5.0$      & $< 6.0$      & $< 7.0$     \\
\hline
no                     & 92.8\%, 6.4M   & 92.8\%, 2.9M   & 92.7\%, 2.8M & 92.7\%, 2.6M & 92.6\%, 2.1M \\
yes                    & 96.3\%, 5.5M   & 96.3\%, 1.2M   & 96.3\%, 0.8M & 96.2\%, 0.2M & 96.2\%, 0.2M \\
\end{tabular}
\caption{Result of applying different pruning strategies on Turku Dependency Treebank models.}
\end{center}
\end{table}

\begin{table}[htb!]
\begin{center}
\includegraphics[width=\textwidth]{pruning}
\end{center}
\end{table}

%\section{Lemmatizer}

%Remember to check how often the correct edit script exists.

%Test leaving out the word form as a feature.