\chapter{Morphology}

\section{The Structure of Words}
%\begin{itemize}
%\item The morphological system.
%\item Word, morpheme, lemma, stem.
%\item Word classes.
%\item Inflectional categories.
%\end{itemize}

Many linguistc units are rather controversial. What is the definition
of a sentence? Do sentences even exist in spoken language? Is there
such a thing as a phoneme?

The concept of a word is, however, among the among the more
universally accepted. The reason for this may be that there is in fact
a fairly easy and readily applicable test for what is a word: X is a
word if X can be used as an answer to a question. For example
\begin{displayquote}
- You said you saw a what yesterday? \\
- Dog. I saw a dog.  
\end{displayquote}
Although, singular nouns in English usually take a compulsory article,
exceptional contexts where they are uttered without an article are
acceptable for many language users. Therefore, it is plausible to
stipulate that singular nouns are words in English. For most
languages, this criterion can be used to find most words that language
users intuitively group under the concept word. Even this criterion,
however, fails for polysemous languages. Nevertheless, it very useful
for many languages.

Of course, orthographic words are much easier to identify than spoken
language words in many writing systems. Simply look for entities
separated by punctuation and white-space. This, however, does not work
for some languages such as Chinese where word spaces are not indicated
in written text. Because this thesis exclusively deals with written
language and because I have performed experiments on European
languages whose orthographies mark word boundaries, I will gloss over
the difficulties of locating word boundaries although this is an
interesting problem from the points of view of both linguistics and
engineering.

Morphology is the sub-fields of linguistics that investigates the
structure of words. For example the knowledge that appending a suffix
``s'' to a singular English nouns makes it plural is morphological
knowledge. Of course this rule is not entirely correct because some
nouns form plurals in some other way (e.g. ``foot/feet'') and yet
other nouns have no plural form (e.g. ``music'').

\begin{figure}
\begin{center}
\caption{The syntagmatic and paradigmatic axes of language.}\label{fig:lingdim}
\includegraphics[scale=0.8]{ling_dim}
\end{center}
\end{figure}

Because words consist of phonemes or orthographic symbols, the
structure of words cannot be investigated without considering the
phonological system of a language. If we had to form the singular
inessive of the the non-existent Finnish noun ``looka'', we would say
``lookassa'', not ``lookassä'' because vowel harmony, a phonological
co-occurrence restriction, prohibits the latter form.


\section{Languages with Rich Morphology}
\begin{itemize}
\item Typological classification of languages.
\item ``Large label sets''.
\end{itemize}

\section{Morphological Analyzers}
\begin{itemize}
\item Finite-state morphology \citep{Koskenniemi1984}, \citep{Kaplan1994}.
\end{itemize}