\chapter{Lemmatization}
\label{chapter:lemmatization}

In this section, I will present the task of data driven
lemmatization. I will examine different approaches to data driven
lemmatization and present the lemmatizer used in the FinnPos toolkit
\cite{Silfverberg2015}.

A lemmatizer is a system which takes text as input and returns the
lemma of each word in the text. Lexical resources such as dictionaries
or morphological analyzers are very helpful for the lemmatization
task. In fact, lemmatization is often seen as one of the sub tasks of
morphological analysis. Another task which is closely related to
lemmatization is {\it morphological paradigm induction}
\citep{Ahlberg2014}. Here the task is to generate all, or a
selection, of the inflectional forms of a word form. Therefore,
lemmatization is a sub-task of morphological paradigm induction.

I will treat lemmatization as a follow up task to morphological
labeling. That is to say, the lemmatizer has access to the
morphological labels in the text. The morphological label is extremely
important for lemmatization. For example, in Finnish a word ending
``-ssa'' could be a noun or verb form. If it is a noun form it could
be either a nominative or an inessive. All of these analyses
correspond to different lemmas. If the correct morphological label is
known, some of the alternative lemmas can be ruled out.

A morphological analyzer can be used for lemmatization of the words in
a text where words have morphological labels. First, analyze each word
using the morphological analyzer. This produces a set of morphological
labels and associated lemmas. Then simply pick the lemma which is
associated with the correct morphological label.

Problems arise when word forms are not recognized by the morphological
analyzer. There are several approaches to solving these problems. One
approach is to utilize the morphological analyzer (for example a
finite-state analyzer) to produce a guess for a lemma even though the
word form is not recognized. The guess is based on orthographically
similar words which are recognized and lemmatized by the morphological
analyzer. As an example of this approach, see \cite{Linden2009}.

The main advantage of basing a data driven lemmatizer on an existing
morphological analyzer is that large coverage morphological analyzers
model most, if not all, morphotactic and the morphophonological
phenomena that occur in a language. Therefore, it is likely that the
analyzer recognizes a number of similar words in the inflectional paradigm
of an unknown word even though it would not recognize that specific
word form which can be utilized in lemmatization. 

Most existing work on analyzer based lemmatizers has used rather
simple statistical models. For example, \cite{Linden2009} uses plain
suffix frequencies. 

\section{Framing Lemmatization As Classification}

In contrast to lemmatizers based on morphological analyzers,
classifier based lemmatizers \cite{Chrupala2008} are learned from data
without an existing model. The general approach is based on the
observation that word forms can be transformed into lemmas using an
{\it edit script}. For example, the English noun ``dogs'' has the
lemma ``dog''. To convert ``dogs'' into ``dog'' one needs to remove
the suffix ``s''. This is a very simple example of an edit script
which I will denote $[-s \rightarrow \epsilon]$. All edit scripts
cannot be applied to all word forms. For example, the edit script which
removes a final ``s'' cannot be applied on past English participle
form ending ``ed''.

Classifier based lemmatizers frame the lemmatization task as a as
classification task. The lemmatizer will use edit scripts as
labels. Subsequently to labeling a word form with an edit script
class, the lemmatizer will apply the edit script thus giving a lemma.

The advantage with using a classifier based lemmatizer is that the
classifier can use a feature based discriminative model. In contrast
to analyzer based lemmatizers, classifier based lemmatizers can
therefore use richer information sources such as prefixes and word
shapes expressed as regular expressions \footnote{An example of a word
  shape expressions in POSIX syntax is {\tt [A-Z][a-z]+} which matches
  capitalized English words.} -- not exclusively information about
word suffixes.

Although it would be very interesting to combine these approaches, it
falls beyond the scope of this thesis. Therefore, I have used
classifier based lemmatizers. I decided upon classifier based
lemmatizers partly because the work of \cite{Linden2009} already
investigates analyzer based lemmatization for Finnish. When performing
morphological disambiguation based on the output of a morphological
analyzer, the current system does use the morphological analyzer for
lemmatization of all word forms which it recognizes. For all remaining
words, the data driven lemmatizer is used.

In the field of morphological paradigm generation, there exists work
which in a sense combines the analyzer and classifier based approaches, for example
\citep{Hulden2014}. However, their starting point is not a morphological
analyzer. Instead a list of morphological paradigms is used. It would
be interesting to explore this but it falls beyond the scope of the
current work.

Joint tagging and lemmatization has also been explored and yields some
improvements \cite{Muller2015}.

%\begin{itemize}
%\item Classifier based: \cite{Chrupala2008}.
%\item Finite-state based: \cite{Linden2009}.
%\item Combination: \cite{Hulden2014}.
%\end{itemize}

A classification based lemmatizer reads in an input form, identifies
the set of edit scripts that can be applied to the input form and
scores the candidate scripts using the input form, its morphological
label and a feature based classifier. Finally, the winning edit script
is applied on the input form and the lemma is recovered.

\paragraph{Extracting Edit Scripts} Given a word form such as ``dogs''
and its lemma ``dog'', several edit scripts can be extracted. For
example, $[-s \rightarrow \epsilon]$, $[-gs \rightarrow -g]$, $[-ogs
\rightarrow -og]$. The current system extracts the shortest script
which adequately recovers the lemma. 

The FinnPos system only extracts edit scripts which delete a suffix
and appends another suffix such as the script $[-s \rightarrow
\epsilon]$. This is mostly sufficient for Finnish all words except numerals
exclusively exhibit inflection at the end of words. Naturally, this would
not be sufficient in general. More general edit scripts hav therefore been used,
for example by \cite{Chrupala2008}.

For morphologically complex languages, a large number of edit
scripts may be extracted from training data. For example, Finnpos
system extracts 4835 different edit scripts for the 145953 tokens in
the training and development data of FinnTreeBank. Therefore, many of
the classes occur few times in the training data. This leads to
data sparsity. However, increasing the amount of the training data
would probably alleviate the problem significantly because the
inventory of inflectional paradigms is finite.

\paragraph{Features for Lemmatization} For a word $w = (w_1...w_n)$ and
a morphological label $y$, the lemmatizer in the FinnPos system
currently uses the following feature templates:
\begin{itemize}
\item The word form $w$.
\item The morphological label $y$.
\item Suffixes $(w_n)$, $(w_{n-1}w_n)$, ... Up to length 10.
\item Prefixes $(w_1)$, $(w_1w_2)$, ... Up to length 10.
\item Infixes $(w_{n-2}w_{n-1})$, $(w_{n-3}w_{n-2})$ and $(w_{n-4}w_{n-3})$.
\end{itemize}
For each feature template $f$ (except the morphological label template
$y$), FinnPos additionally uses a combination template $(f,y)$ which
captures correlations between morphological labels and the
orthographical representation of the word form.

The infix templates are useful because they model the environment of
where an inflectional suffix like ``-s'' is removed and a lemma suffix
is added. They aim at preventing phonotactically impossible
combinations.

\paragraph{Estimation} The lemmatizer can be implemented using any
discriminative classifier. For example as an averaged perceptron
classifier or a logistic classifier. In the FinnPos system, the
lemmatizer is an averaged perceptron classifier.

The estimation of the lemmatizer model differs slightly from standard
averaged perceptron estimation presented in Chapter
\ref{chap:ml}. Even though the number of edit scripts can be very
large (in the order of thousands), the subset of edit scripts
applicable for any given word form is much smaller. Moreover, it is
always known in advance because it is completely determined by the
suffixes of the word form. Therefore, the classifier is only trained
to disambiguate between the possible edit scripts associated to each
word form. This speeds up estimation considerably.

\paragraph{Inference} In the FinnPos system, words which were seen
during training time, are lemmatized based on a lemma dictionary which
associates each pair of word form and morphological label with a
lemma. For words which were not seen during training or which received
a label not seen during training, are lemmatized using the data driven
lemmatizer. Additionally, a morphological analyzer can be used to
assign lemmas to those words which it recognizes.

For word forms which cannot be lemmatized using the lemma dictionary
or morphological analyzer, the data driven lemmatizer is used. For
each word form, the set of applicable edit scripts is formed and
scored. The highest scoring edit script is subsequently applied to the
word form to produce a lemma.

