\documentclass[b5paper]{article}

\usepackage{polyglossia}
\usepackage{fontspec}
\usepackage{xunicode}
\usepackage{xltxtra}
\usepackage{url}
\usepackage{hyperref}
\usepackage{expex}

\makeatletter
\def\blfootnote{\gdef\@thefnmark{}\@footnotetext}
\makeatother

\setmainfont[Mapping=tex-text]{Linux Libertine O}

\begin{document}

\title{Can Morphological Analyzers Improve the Quality of Optical Character Recognition?}

\author{Miikka Silfverberg\\
University or Helsinki\\
Dept. of Modern Languages\\
\\
\url{mpsilfve@iki.fi} \and
Jack Rueter\\
University or Helsinki\\
Dept. of Modern Languages\\
\\
\url{jack.rueter@helsinki.fi} 
}

\date{\today}

\maketitle

\begin{abstract}
  {\it Optical Character Recognition} (OCR) can subtantially improve
  the usability of digitized documents. Language modelling using word
  lists is known to improve OCR quality of English. However, for
  morphologically rich languages, word lists do not reach high
  coverage on unseen text. Morphological analyzers offer a more
  sophisticated approach, which is useful in many language processing
  applications. This paper investigates language modelling in the
  open-source OCR engine Tesseract using morphological analyzers. We
  present experiments on two Uralic languages Finnish and
  Erzya. According to our experiments, word lists may still be
  superior to morphological analyzers in OCR even for highly inflected
  languages.
\end{abstract}

\section{Introduction}

Digital media is an integral part of modern society and digitization
of print matter is crucial for the viability of minority languages. It
also serves the linguistic community by making print media widely
available. However, simply scanning documents is not enough, because
few applications can deal with images directly. {\it Optical Character
  Recognition} (OCR) can subtantially improve the usability of
digitized documents for example by allowing search engines to index
the documents. In this paper, we investigate improving the quality of
OCR for languages with rich morphology.

OCR engines can benefit from language modelling, which is a field
encompassing a variety of techniques that aim at improving the
function of language processing applications by capturing key
properties of the target language. Among others, translation software
and speech recognizers benefit greatly from sophisticated language
models. In OCR simpler language models can be adequate. Often word
lists are used.

Word lists are adequate in applications designed for languages with
limited morphology such as English. However, morphologically rich
languages, among others the Uralic languages, require more elaborate
approaches. For these languages, even extensive word lists are
unlikely to reach very high coverage on previously unseen text because
of extensive compounding, derivation and inflection \cite{creutz07}.

In contrast to word lists, a morphological analyzer
\cite{koskenniemi83}, which encodes the derivational and inflectional
morphology of a language, can achieve substantially higher
coverage. Thus it is conceivable that language models utilizing
morphological analyzers could substantially improve the quality of OCR
for morphologically rich languages.

In this paper, we present experiments on OCR for two Uralic languages,
Finnish and Erzya. We utilized the open-source OCR engine Tesseract
\cite{smith07} together with two open-source morphological analyzers
for Erzya and Finnish respectively.

In light of our experiments, it seems that morphological analyzers do
help in OCR of morphologically rich languages, but the improvements
are modest over a baseline model without a language model. Moreover,
we were unable to get improvements upon using word lists harvested
from the Wikipedia data bases for Erzya in Finnish. Error analysis
revealed that the high coverage of the morphological analyzer may in
fact present problems for the OCR process, as it leads to a
substantial increase in real word errors compared to word lists.

Although we did not get improvements over word lists, it is worth
pointing out that for some under-resourced minority languages
morphological analyzers created by linguists represent the best
existing lexical resources in digital format. The reason for this is
that digital content on the internet can be very scarce and the
orthography may be non-standard. Therefore, using a morphological
analyzer as part of an OCR engine can still be motivated.

The paper is structured as follows. We describe related work in
Section \ref{rel}. In Section \ref{met} we describe the Tesseract OCR
engine, the morphological analyzers and their integration. Section
\ref{exp} details the experimental setup. In Section \ref{res}, we
present the results of the expeirments. We present a brief error
analysis of the experiments on Finnish in Section \ref{err} and
conclude the paper in Section \ref{disc}.

\section{Related Work}
\label{rel}
To our knowledge, this is the first investigation into using
morphological analyzers in OCR. However, other approaches to language
modelling for languages with extensive inflection have been investigated.

Smith et. al \cite{smith09} add a module, which expands the vocabulary
by generating word forms from stem suffix pairs. Their method requires
no additional linguistic resources, since the sets of stems and
affixes are inferred from the vocabulary. However, we believe this
approach is unlikely to work well with extensive compounding present
in for example Finnish, because of data sparseness.

There is a large body of literature on spelling correction for
inflecting languages e.g. \cite{oflazer94} and \cite{pirinen12} to name a
few. These approaches have been successfully applied to OCR
post-processing. In our work, we wanted to investigate changing the
language model directly instead of applying post-processing, as we
felt that post-processing cannot in principle give as good
results. The motivation for this is that knowledge about the
reliability of predictions of the individual characters has already
been lost at the post-processing stage.

Lastly, character based statistical language models have been
investigated, but the results of this approach are not encouraging
\cite{smith11}. It seems that statistical language models may in fact
degrade the performance of high accuracy OCR engines. Nevertheless,
this approach has given good results in the related field of
handwritten text recognition \cite{marti01}, where the overall
performance is much lower.

\section{Methods}
\label{met} In this section, we describe the Tesseract OCR engine, the
HFST finite-state library as well as HFST morphological analyzers, and
the process of combining these utilities.

\subsection{Tesseract}
The Tesseract\footnote{https://code.google.com/p/tesseract-ocr/} OCR
engine \cite{smith07} was originally developed at HP Labs between 1984
and 1995 for high quality OCR for English. In 2005 it was released as
an open-source project and has since been applied to several languages
and alphabets. Tesseract was therefore a natural choice for exploring
improvements for OCR of Uralic languages.

The recognition process of Tesseract can be seen as a pipeline
consisting of four phases \cite{smith07}:
\begin{enumerate}
\item Identification of character boundaries.
\item Grouping of characters into words and lines.
\item Word Recognition.
\item Resolution of ambiguous word spacing.
\end{enumerate}
Phase three is in fact run twice, since Tesseract uses an adaptive
classifier, which learns from the text at the same time as it
classifies.

Our work focuses on the third phase of the pipeline, namely word
recognition. Word recognition encompasses two sub-tasks: character
recognition using a neural network and word level classification using
a second adaptive classifier and various language models. During word
recognition, the adaptive classifier and language models give
competing suggestions based on the output of character recognizer.

We do not modify the character level model in any way. Instead, we
replace the existing language model of Tesseract with a morphological
analyzer.

The existing language models in Tesseract are word lists, which are
compiled into {\it directed acyclic graphs} (DAG) for fast
processing. Tesseract utilizes a number of different language
models\footnote{https://code.google.com/p/tesseract-ocr/wiki/TrainingTesseract3},
for example:
\begin{itemize}
\item Most frequent word forms.
\item Dictionary word forms.
\item Punctuation patters.
\item Words containing digits.
\end{itemize}

Each language model and the adaptive classifier have an associated
weight. When the character model returns a distribution over possible
word forms corresponding to an input, the language models are used to
re-weight the distribution based on distance from a word in the model
and the model weight. For example, the frequent word model has a
greater weight than the dictionary model. Therefore, its suggestions
are more likely to be accepted.

\subsection{Helsinki Finite-State Technology}
Helsinki Finite-State Technology (HFST) \cite{linden13} is an
open-source C++ library for constructing finite-state
transducers. Among other utilities, the HFST library contains tools
for construction of morphological analyzers. Morphological analyzers
compatible with the HFST library exist for several languages. We know
of at least fifteen Uralic
languages\footnote{http://giellatekno.uit.no/all-lang.eng.html} with
HFST morphological analyzers among those Erzya and Finnish.

\subsection{Morphological Analyzers as OCR Language Models}
As mentioned above, Tesseract internally represents language models as
directed acyclic graphs or DAGs. HFST morphological analyzers are
finite-state transducers (FST), which are closely related to DAGs. The
main difference is that finite-state transducers transform strings
instead of simply accepting or discarding them. Additionally,
finite-state transducers can be cyclic unlike DAGs.

There exists a straight-forward conversion ({\it projection})
from FSTs to finite-state automata, which are identical to DAGs in
other respects, but may be cyclic like FSTs. 

It turned out, that Tesseract is not in principle incompatible with
cyclic graphs. The existing implementation simply did not offer a way
to produce cyclic graphs. Fortunately, it was not difficult to
implement a sub-class for the Tesseract language modelling class, {\tt
  dict}, which does support cyclic graphs. Additionally, we
implemented a driver for HFST transducers in optimized lookup format,
which supports lookup speeds of up-to 100 000 words/s
\cite{silfverberg09}.

HFST morphological analyzers can contain so called flag diacritics
\cite{beesley03}, which are used to compress the finite-state machine
by introducing non-determinism in a controlled way. Tesseract employs
a Viterbi-like algorithm for finding word suggestions, which requires
the language model to be deterministic. Hence it cannot handle flag
diacritics. However, HFST includes utilities which can be used to
eliminate the flag diacritics from a finite-state machine without
changing its behaviour. We did this.

All the necessary steps to transform an HFST morphological analyzer
into a Tesseract language model have been incorporated into the HFST
interface as the tool\\{\tt hfst-fst2tesseract}.

\section{Experimental Setup}
\label{exp}

In this section we describe the data sets used in the experiment, the
evaluation scheme and the experiments.

\subsection{Data}
We evaluate the impact of morphological analyzers in OCR for two
Uralic languages, Erzya and Finnish. For both languages, we perform
experiments on excerpts from novels.
 
The Erzya language has a relatively rich morphological system of
regular inflection, most extensive in the verbs and nouns. The verbs
attest to object and subject conjugation in 7 moods, whereas there are
9 declensions for 9-15 regular case forms in nouns, with additional
conjugation possibilities in two tenses for 3-6 of those. Erzya is
written using the Cyrillic alphabet.

Finnish is similar to Erzya in that it has a extensive noun and verb
inflection. Additionally, Finnish has a productive compounding
mechanism, which gives rise to an extensive vocabulary. Unlike Erzya,
Finnish is written using the Latin alphabet.

For Finnish, we use pages 5 - 21 of the novel Elokuu by
F.E. Sillanp\"{a}\"{a} \cite{sillanpaa08} (3219 tokens, 24096 characters) and for Erzya, we use pages
3 - 21 of the novel XX by Maksim Gorki \cite{gorki} (4539 tokens, 58548 characters). In order to gauge
the effect of different language models on scanned material of varying
quality, the data were scanned in three different resolutions: 100
dpi, 200 dpi and 300 dpi.

Using correctly trained model with no vocabulary, standard Tesseract
performs adequately on scanned images of quality 300 dpi. The results
require little manual correction. However, 100 dpi usually gives quite
poor performance. In fact the performance is so poor that manual
correction might take longer than simply writing the text.

\subsection{Linguistic Resources}
For constructing Tesseract models with vocabulary, we used the text
dumps of the
Erzyan\footnote{\url{ftp://wikipedia.c3sl.ufpr.br/wikipedia/myvwiki/20140927/myvwiki-20140927-pages-meta-current.xml.bz2}}
and
Finnish\footnote{\url{ftp://wikipedia.c3sl.ufpr.br/wikipedia/fiwiki/20141018/fiwiki-20141018-pages-meta-current.xml.bz2}}
Wikipedias. We used xml containing the current versions of all
articles in the Wikipedias of the respective languages. For extracting
the text contents, we used the utility {\tt
  wp2text}~\footnote{\url{https://github.com/yohasebe/wp2txt}}.

In addition to Wikipedia text, we used freely available morphological
analyzers for Finnish and Erzya. OMorFi \cite{pirinen11} is a broad
coverage Finnish morphological analyzer available
online\footnote{\url{https://code.google.com/p/omorfi/}}. For Erzya,
we used the Erzyan analyzer distributed by the Giellatekno project
\cite{moshagen14}.

The coverages of different linguistic resources on test data are shown
in Table \ref{fin-myv-coverage}. For both languages, the coverage is
best using the morphological analyzer. However, for Finnish, the
coverage of the one million word list is very close to the coverage of
the analyzer.

\begin{table}[!htb]
\begin{center}
\begin{tabular}{lr}
\hline 
                        & Coverage  \\
\hline 
1K word list            &   28.0\%  \\
10K word list           &   49.5\%  \\
68K word list           &   58.6\%  \\
Morphological analyser  &   80.6\%  \\
                        &
\end{tabular}
\quad
\begin{tabular}{lr}
\hline 
                        & Coverage \\
\hline 
1K word list            &  32.2\% \\
10K word list           &  52.8\% \\
100K word list          &  71.4\% \\
1000K word list         &  84.5\% \\
Morphological analyser  &  86.7\% 
\end{tabular}
\caption{Coverages of linguistic resources on Erzya and Finnish test material.}\label{fin-myv-coverage}
\end{center}
\end{table}

\subsection{Experiments}
We trained six different OCR systems both for Finnish and four models for Erzya:
\begin{itemize}
\item A system without a language model (the baseline).
\item Systems using 1000 and 10 000 word vocabularies for both Finnish
  and Erzya, a 68000 word system for Erzya and 100 000 and 1 million
  word systems for Finnish.
\item A system using a morphological analyzer as language model.
\end{itemize}
We could not use more than 68000 words for Erzya as the Erzya
Wikipedia does not contain more unique word forms.

For Finnish, we constructed the baseline system simply by deleting the
vocabularies ({\tt freq-dawg} and {\tt word-dawg}) from the existing
Tesseract OCR system for Finnish~\footnote{see:
  \url{https://code.google.com/p/tesseract-ocr/downloads/list}}. For
Erzya, we trained our own baseline model.

In order to compile systems with vocabularies ranging from 1000 words
to 1 million words, we extracted the most common N words from the
Wikipedia, compiled them into a directed acyclic graph using the
Tesseract utility {\tt wordlist2dawg} and used the graphs as word the
model ({\tt word-dawg}) in a system which otherwise was identical to
the baseline system.

The morphological analyzers, were first processed using the HFST
utility\\{\tt hfst-fst2tesseract}. We then combined them with the
baseline systems.

\subsection{Evaluation}
It is tempting to view OCR as a special case of sequence labelling,
since an OCR engine essentially labels the characters in a digitized
text using their alphabetical correspondents. This resembles the
prototypical sequence labelling task part-of-speech tagging, where
words are labelled using part-of-speech tags. This similarity suggest
evaluation based on character error rate in relation to a gold
standard text. However, the parallel between an OCR engine and
part-of-speech tagger is not entirely accurate.

Simple metrics such as character level accuracy cannot be used, since
OCR frequently changes the length of the underlying text, because
spurious characters may be inserted and characters in the text may be
deleted. Therefore, we evaluated by measuring the {\tt edit distance}
of the OCR result and the gold standard.

In practice, we first aligned the text on character level using the
Unix utility {\tt diff}. We then computed the number of edits required
to transform the OCR result into the gold standard text. We call this
figure the {\it edit count} (EC). For each experiment, we report both
raw edit counts and the reduction in edit count (ER) compared to the
baseline OCR system without language modelling.

The edit reduction, from a baseline edit count $B$ to an improved edit
count $C$ is defined as
$${\rm ER} = \frac{B - C}{B}$$
If, the baseline $B$ is in fact better than the count $C$, the edit
reduction will be be negative.

We performed paired one sided Wilcoxon test tests to asses the
statistical significance of our results. We compared all models to the
baseline model. We additionally compared the best word-list system to
the sytem using a morphological analyzer.

\section{Results}
\label{res}
In this section we show the results for Finnish in Table
\ref{fin-novel-res} and for Erzya in Table
\ref{myv-novel-res}. 

For the Finnish novel, all models utilizing some kind of language
modelling faired better than the baseline model without any kind of
vocabulary information. The morphological analyzer performed better
than the other models on the lowest image quality 100 dpi. Otherwise,
it in fact performed worse than all other models utilizing language
modelling.

For resolutions 300 and 200 dpi, all language models gave
statistically significant improvements over the baseline in the 95\%
confidence interval. The best word list model was better than the
morphological analyzer. For 100 dpi, only the morphological analyzer
performed better than the baseline in the 95\% confidence interval.

\begin{table}[!htb]
\begin{center}
\begin{tabular}{lrrr}
\hline 
                  & 300 dpi & 200 dpi & 100 dpi \\
\hline 
No language model & ~0.0\% (794)          & ~0.0\% (1265)          & 0.0\% (15504)  \\
1000 words        & ~32.1\% (539)  & ~36.8\% (799)        & 2.1\% (15172)           \\
10 000 words      & {\bf ~35.3}\% (514)  & ~44.7\%  (699)  & 4.0\% (14891)          \\
100 000 words     & ~31.5\% (544)   & ~44.0\%  (708)  & 3.2\%  (15014)              \\
1 million words   & ~33.5\% (528)   & {\bf ~45.4\%} (691)  & 2.4\% (15131)          \\
Morph. analyzer   & ~25.3\% (593)    & ~30.0\% (885)     & {\bf 5.7\%} (14621)      \\
\hline 
\end{tabular}
\caption{Edit reduction (and total edit count) for the Finnish novel Elokuu using different models and resolutions.}\label{fin-novel-res}
\end{center}
\end{table}

The results for Erzya parallel those of Finnish. The morphological
analyzer improves over the word lists only forse the lowes resolution
100 dpi. For resolution 200 dpi, the morphological analyzer does not
seem to have any effect.

\begin{table}[!htb]
\begin{center}
\begin{tabular}{lrrr}
\hline 
                  & 300 dpi & 200 dpi & 100 dpi \\
\hline 
No language model &  0.0\% (3257)  &  0.0\% (3224)  &  0.0\% (15788)  \\
1000 words        &  20.9\% (2576)  &  11.7\% (2846)  & -10.7\%  (17473) \\
10 000 words      &  29.5\% (2295)  &   {\bf 22.7\%} (2492)  & 1.8\% (15498)  \\
68K words         &  {\bf 30.9\%} (2249)  &  21.9\% (2517)  & 0.5\% (15702)\\
Morph. analyzer   &  8.0\% (2996)  &  -0.1\% (3230)  & {\bf 2.8\%} (15353)  \\
\hline 
\end{tabular}
\caption{LER for the Erzyan novel FOO using different models and resolutions.}\label{myv-novel-res}
\end{center}
\end{table}

\section{Error Analysis}
\label{err}

In this section, we present a comparison of the OCR errors produced by
the Finnish systems utilizing a morphological analyzer and the 1000K
word list as language models. We only examine the errors for the
highest image quality 300 dpi.

We classify errors into classified into two types: real word errors
and other errors. Real word errors are errors, where the resulting
word is known by the language model, for example a clitizized genitive
of he/she``Hänenkin'' is recognized erroneously as a clitizized
genitive of pike ``Hauenkin''. Other errors simply encompass all other
error types, common examples include insertion and deletion of
punctuation and casing errors.

A total of 18\% of the errors produced by the morphological analyzer
are real word errors. In contrast the word list only gives 2\% real
word errors.

\section{Discussion and Conclusions}
\label{disc}
Doesn't seem to be hugely influential :/
\section*{Acknowledgments}

...

\bibliographystyle{unsrt}
\bibliography{fiwclul2015}

\end{document}

