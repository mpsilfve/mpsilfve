\documentclass[b5paper]{article}

\usepackage{polyglossia}
\usepackage{fontspec}
\usepackage{xunicode}
\usepackage{xltxtra}
\usepackage{url}
\usepackage{hyperref}
\usepackage{expex}

\makeatletter
\def\blfootnote{\gdef\@thefnmark{}\@footnotetext}
\makeatother

\setmainfont[Mapping=tex-text]{Linux Libertine O}

\begin{document}

\title{Can Morphological Analyzers Improve the Quality of Optical Character Recognition?}

\author{Miikka Silfverberg\\
University or Helsinki\\
Dept. of Modern Languages\\
\\
\url{mpsilfve@iki.fi} \and
Jack Rueter\\
University or Helsinki\\
Dept. of Modern Languages\\
\\
\url{jack.rueter@helsinki.fi} 
}

\date{\today}

\maketitle

\begin{abstract}
    Abstract can be between 150 and 300 words long. Separate keywords are not
    used. Reviewers use abstract to decide which articles to review and readers
    will use abstract to decide which articles to read, so make it informative
    and interesting.
\end{abstract}

\section{Introduction}

Digital media is an integral part of modern society and digitization
of print matter is crucial for the viability of minority
languages. Digitization also serves the linguistic community by making
print media widely available. 

{\it Optical Character Recognition} (OCR) can subtantially improve the
usability of digitized documents by allowing for search in
the document. In this paper, we investigate improving the quality of
OCR using {\it morphological analyzers}.


Languages modelling is a field encompassing a variety of
techniques that aim at improving the function of language applications
such as translation software and speech recognizers by capturing key
properties of the target language. In OCR, language modelling usually
amounts to using lista of word forms.

Word lists are often adequate language models in applications designed
for languages with limited morphology, for example English. However,
morphologically rich languages, such as the Uralic languages, require
more elaborate approaches. Even extensive word lists are unlikely to
reach very high coverage on previously unseen text [CITE] because of
extensive compounding, derivation and inflection.

In contrast, a morphological analyzer [CITE], which encodes the
derivational and inflectional morphology of a language, can achieve
substantially higher coverage. Thus it is conceivable that morphological
analyzers could substantially improve the quality of OCR for
morphologically rich languages.

We present experiments on OCR for two Uralic languages, Finnish and
Erzya. In this paper, we utilize the open-source OCR engine Tesseract
\cite{smith07} and the morphological analyzers OMorFi \cite{pirinen11}
for Finnish and FOO for Erzya [CITE].

In light of our experiments, it seems that morphological analyzers do
help in OCR of morphologically rich languages, but the improvements
are modest.

Nevertheless, it is worth pointing out, that for under-resourced
minority languages such as Erzya, morphological analyzers created by
linguists may represent the best existing lexical resources in digital
format. The reason for this is that digital content on the internet
can be very scarce. Therefore, using a morphological analyzer as part
of an OCR engine can be superior to using a word list simply because
no sufficiently broad coverage word lists exist.

\section{Related Work}
\begin{itemize}
\item \cite{smith09} add a module, which generates missing word
  forms. This approach is unlikely to work well with compounds.

\end{itemize}

\section{Methods}
\begin{itemize}
\item We modified the language modelling component of the Tesseract
  OCR engine so that it could use HFST finite-state transducers in
  optimized lookup format \cite{silfverberg09} as language models.
\end{itemize}

\subsection{Tesseract}
\begin{itemize}
\item Language modelling in Tesseract consists of a number of word
  lists, e.g. a list of frequent word and words containing digits
  \cite{smith07}.
\item The word lists are compiled into directed acyclic graphs (DAWG)
  that are in essence acyclic finite-state acceptors. Each acceptor
  receives a global weighting coefficient, which determines its
  reliability.
\end{itemize}
\subsection{Helsinki Finite-State Technology}
\begin{itemize}
\item HFST \cite{linden13}
\item It is quite easy to integrate a morphological analyzer into a
  scenario where dictionaries are already represented as DAWGs.
\item At the surface the analyzer, converted into a cyclic
  finite-state acceptor, does not differ from the word lists in DAWG
  format. However it does function differently, because it is capable
  of recognizing an infinite language.
\item The analyzer receives a weighting coefficient.
\item Optimized lookup format is fast.
\item Memory consumption goes up, but not prohibitively.
\end{itemize}

\subsection{Using a Morphological Analyzer as an OCR Language Model}

\section{Experiments}

%\begin{itemize}
%\item For both languages, we compare the following models:
%\begin{enumerate}
%\item No language model.
%\item Dictionary of word forms of varying sizes (1K, 100K and 1000K).
%\item A morphological analyzer.
%\end{enumerate} 
%
%\item The dictionaries were composed of the most frequent N words in
%  the Wikipedias of the respective language. It is clear that the
%  choice of training material and especially the domain can influence
%  the results. Wikipedia should cover several different domains.
%
%\item The test set consists of standard prose.
%
%\item We created a hand crafted gold standard of 20 pages for each
%  language.
%
%\item Evaluation metric: Letter Edit Rate (LER). Amount of edits / maximal
%  amount of edits. The metric corresponds well to the amount of manual
%  post processing required [CITE].
%
%\item The OCR result and gold standard are first aligned using
%  approximate edit distance (implemented by the unix utility diff),
%  then the number of edits is computed from the aligned texts.
%\end{itemize}
In this section we describe the experimental setup of the paper.

\subsection{Data}
We evaluate the impact of morphological analyzers in OCR for two
Uralic languages, Erzya and Finnish. For both languages, we perform
experiments on excerpts from a novel and from newspaper text.
 
For Finnish, we use pages 5 - 21 of the novel Elokuu by
F.E. Sillanp\"{a}\"{a} \cite{sillanpaa08}. Additionally, we use the
pages XX-YY of the Helsingin Sanomat newspaper from XX.YY.ZZZZ.

For Erzya, we use pages 3 - 21 of the novel XX by Maksim Gorki
\cite{gorki}. Additionally, we use pages XX - YY on September 11th 1927
from the Erzyan newspaper Од эрямо.

In order to gauge the effect of different language models on scanned
material of varying quality, the data were scanned in three different
resolutions: 100 dpi, 200 dpi and 300 dpi.

Using correctly trained model with no vocabulary, standard Tesseract
performs adequately on scanned images of quality 300 dpi. The results
require little manual correction. However, 100 dpi usually gives quite
poor performance. In fact the performance is so poor that manual
correction might take longer than simply writing the text.

\begin{table}[!htb]
\begin{center}
\begin{tabular}{lrr}
\hline 
        & Novel~~~~~~~~~~~~~~~~~~   & Newspaper \\
\hline 
Finnish & 3219 tokens (24096 characters)  &          \\
Erzya   & 4539 tokens (58548 characters)  &          \\
\hline 
\end{tabular}
\caption{Numeric description of data sets.}\label{data-table}
\end{center}
\end{table}

\subsection{Linguistic Resources}
For constructing Tesseract models with vocabulary, we used the text
dumps of the Erzyan and Finnish Wikipedias. We used
xml-files~\footnote{\url{www.foo.bar/baz} and \url{www.foo.bar/baz}}
containing the current versions of all articles in the Wikipedias of
the respective languages. For extracting the text contents, we used
the utility {\tt wikipedia2text}~\footnote{\url{www.foo.bar/baz}}.

In addition to Wikipedia text, we used freely available morphological analyzers for Finnish and Erzya. OMorFi \cite{pirinen11} is a broad coverage Finnish morphological analyzer available online\footnote{\url{https://code.google.com/p/omorfi/}}. For Erzya, we used the Erzyan analyzer distributed by the Giellatekno project \cite{moshagen14}.

The coverages of different linguistic resources on test data are shown
in Tables \ref{fin-coverage} and \ref{myv-coverage}.

\begin{table}[!htb]
\begin{center}
\begin{tabular}{lrr}
\hline 
                        & Novel   & Newspaper \\
\hline 
1K word list            &  32.2\% &         - \\
10K word list           &  52.8\% &         - \\
100K word list          &  71.4\% &         - \\
1000K word list         &  84.5\% &         - \\
Morphological analyser  &  86.7\% &         - \\
\hline 
\end{tabular}
\caption{Coverages of linguistic resources on Finnish test material.}\label{fin-coverage}
\end{center}
\end{table}

\begin{table}[!htb]
\begin{center}
\begin{tabular}{lrr}
\hline 
                        & Novel     & Newspaper \\
\hline 
1K word list            &   28.0\%  &         - \\
10K word list           &   49.5\%  &         - \\
68K word list           &   58.6\%  &         - \\
Morphological analyser  &   80.6\%  &         - \\ 
\hline 
\end{tabular}
\caption{Coverages of linguistic resources on Erzya test material.}\label{myv-coverage}
\end{center}
\end{table}

\subsection{Methods}
We trained six different models for both Finnish and Erzya.
\begin{itemize}
\item A model without a language model.
\item Models with a 1000, 10 000, 100 000 and 1 million word vocabularies respectively.
\item A model using a morphological analyzer as language model.
\end{itemize}

For Finnish, we constructed the model without language model by
deleting the vocabularies ({\tt freq-dawg} and {\tt word-dawg}) from
the existing Tesseract model for Finnish~\footnote{see:
  \url{https://code.google.com/p/tesseract-ocr/downloads/list}}.

In order to compile the models with vocabularies ranging from 1000
words to 1 million words, we extracted the most common N words from
the Wikipedia, compiled them into a directed acyclic graph using the
Tesseract utility {\tt wordlist2dawg} and used the graphs as word
model ({\tt word-dawg}).

DESCRIBE ANALYZER MODELS.

\subsection{Evaluation}
From the point-of-view of evaluation, it is tempting to view OCR as a
special case of sequence labelling. In sequence labelling tasks, such
as part-of-speech tagging, words are labelled using part-of-speech
labels from a set of possible labels. Similarly, an OCR engine labels
the character glyphs in an image using characters from the
alphabet. This similarity suggests evaluation using character level
accuracy. However, the parallel between an OCR engine and
part-of-speech tagger is not entirely accurate.

Simple metrics such as character level accuracy cannot be used, since
the OCR process gives rise to error types, which change the length of
the text. We classify OCR errors into the following categories:
\begin{center}
\begin{tabular}{ll}
Deletions  & Characters vanish. E.g. ``dog'' is recognized as ``dg''\\
Insertions & Spurious characters emerge. E.g. noise in the image may \\
           & be recognized as ``il-\_.:''.\\ 
Swaps      & Characters may be swapped for other characters. E.g. \\
           & ``dog'' can be recognized as ``d0g''.
\end{tabular}
\end{center}

Because the OCR result an the gold standard text can differ
significantly in length, we first align on character level using the
unix utility {\tt diff}. We then compute the number of edits required
to transform the OCR result into the gold standard text and call this
number the {\it edit count} (EC).

For each experiment, we report both raw edit counts and the reduction
in edit count (ER) when compared to a baseline OCR model. The edit
reduction, when comparing edit count $C$ to a basline edit count $B$,
is defined as
$${\rm ER} = \frac{B - C}{B}$$
If, the baseline $B$ is in fact better than the count $C$, the edit
reduction can be negative.

We performed statistical significance tests to asses which of the
models faired better thand the baseline. Additionally we tested the
difference between the best model with a word list compared to the
model using a morphological analyzer. We use a paired Wilcoxon test.

\section{Results}

In this section we show the results for Finnish in Tables
\ref{fin-novel-res} and \ref{fin-news-res} and for Erzya in Tables
\ref{myv-novel-res} and \ref{myv-news-res}. 

For the Finnish novel, all models utilizing some kind of language
modelling faired better than the baseline model without any kind of
vocabulary information. The morphological analyzer performed better
than the other models on the lowest image quality 100 dpi. Otherwise,
it in fact performed worse than all other models utilizing language
modelling.

For resolutions 300 and 200 dpi, all language models gave statistically significant improvements over the baseline in the 95\% confidence interval. The best word list model was better than the morphological analyzer. For 100 dpi, only the morphological analyzer performed better than the baseline in the 95\% confidence interval.

\begin{table}[!htb]
\begin{center}
\begin{tabular}{lrrr}
\hline 
                  & 300 dpi & 200 dpi & 100 dpi \\
\hline 
No language model & ~0.0\% (794)          & ~0.0\% (1265)          & 0.0\% (15504) \\
1000 words        & ~32.1\% (539)  & ~36.8\% (799)        & 2.1\% (15172)    \\
10 000 words      & {\bf ~35.3}\% (514)  & ~44.7\%  (699)  & 4.0\% (14891)   \\
100 000 words     & ~31.5\% (544)   & ~44.0\%  (708)  & 3.2\%  (15014)  \\
1 million words   & ~33.5\% (528)   & {\bf ~45.4\%} (691)  & 2.4\% (15131)        \\
Morph. analyzer   & ~25.3\% (593)    & ~30.0\% (885)     & {\bf 5.7\%} (14621)  \\
\hline 
\end{tabular}
\caption{Edit reduction (and total edit count) for the Finnish novel Elokuu using different models and resolutions.}\label{fin-novel-res}
\end{center}
\end{table}

The results for Erzya parallel those of Finnish. The morphological analyzer improves over the word lists only forse the lowes resolution 100 dpi. For resolution 200 dpi, the morphological analyzer does not seem to have any effect.

%\begin{table}[!htb]
%\begin{center}
%\begin{tabular}{lrrr}
%\hline 
%                  & 300 dpi~~~~ & 200 dpi~~~~ & 100 dpi~~~~ \\
%\hline 
%No language model & xx.xx\%~~~~   & xx.xx\%~~~~   & xx.xx\%~~~~   \\
%1000 words        & xx.xx\%~~~~   & xx.xx\%~~~~   & xx.xx\%~~~~   \\
%10 000 words      & xx.xx\%~~~~   & xx.xx\%~~~~   & xx.xx\%~~~~   \\
%100 000 words     & xx.xx\%~~~~   & xx.xx\%~~~~   & xx.xx\%~~~~   \\
%1 million words   & xx.xx\%~~~~   & xx.xx\%~~~~   & xx.xx\%~~~~   \\
%Morph. analyzer   & xx.xx\%~~~~   & xx.xx\%~~~~   & xx.xx\%~~~~   \\
%\hline 
%\end{tabular}
%\caption{LER for the Finnish newspaper excerpt using different models and resolutions.}\label{myv-novel-res}
%\end{center}
%\end{table}

\begin{table}[!htb]
\begin{center}
\begin{tabular}{lrrr}
\hline 
                  & 300 dpi & 200 dpi & 100 dpi \\
\hline 
No language model &  0.0\% (3257)  &  0.0\% (3224)  &  0.0\% (15788)  \\
1000 words        &  20.9\% (2576)  &  11.7\% (2846)  & -10.7\%  (17473) \\
10 000 words      &  29.5\% (2295)  &   {\bf 22.7\%} (2492)  & 1.8\% (15498)  \\
68K words         &  {\bf 30.9\%} (2249)  &  21.9\% (2517)  & 0.5\% (15702)\\
Morph. analyzer   &  8.0\% (2996)  &  -0.1\% (3230)  & {\bf 2.8\%} (15353)  \\
\hline 
\end{tabular}
\caption{LER for the Erzyan novel FOO using different models and resolutions.}\label{fin-news-res}
\end{center}
\end{table}


%\begin{table}[!htb]
%\begin{center}
%\begin{tabular}{lrrr}
%\hline 
%                  & 300 dpi~~~~ & 200 dpi~~~~ & 100 dpi~~~~ \\
%\hline 
%No language model & xx.xx\%~~~~   & xx.xx\%~~~~   & xx.xx\%~~~~   \\
%1000 words        & xx.xx\%~~~~   & xx.xx\%~~~~   & xx.xx\%~~~~   \\
%10 000 words      & xx.xx\%~~~~   & xx.xx\%~~~~   & xx.xx\%~~~~   \\
%100 000 words     & xx.xx\%~~~~   & xx.xx\%~~~~   & xx.xx\%~~~~   \\
%1 million words   & xx.xx\%~~~~   & xx.xx\%~~~~   & xx.xx\%~~~~   \\
%Morph. analyzer   & xx.xx\%~~~~   & xx.xx\%~~~~   & xx.xx\%~~~~   \\
%\hline 
%\end{tabular}
%\caption{LER for the Erzyan newspaper excerpt using different models and resolutions.}\label{myv-news-res}
%\end{center}
%\end{table}

\section{Error Analysis}
foo
 
\section{Discussion and Conclusions}

Doesn't seem to be hugely influential :/
\section*{Acknowledgments}

...

\bibliographystyle{unsrt}
\bibliography{fiwclul2015}

\end{document}

